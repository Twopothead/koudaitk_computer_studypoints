\textbf{{1.磁盘的物理结构}}\\

磁盘上的一系列同心圆称为磁道,磁道沿径向又分成大小相等的多个扇区,盘片上与盘片中心有一定距离的所有磁道组成了一个柱面。因此,\textbf{磁盘上的每个物理块可以用柱面号、磁头号和扇区号表示}。

\textbf{{2.磁盘结构中的信息}}

a. 引导控制块:通常为分区的第一块,如果该分区没有操作系统,则为空;\\
b.
分区控制块:包括分区的详细信息,如分区的块数、块的大小、空闲块的数目和指针等;\\
c. 目录结构:采用目录文件组织;\\
d.
文件控制块:包括文件的信息,如文件名、拥有者、文件大小和数据块位置等。

\textbf{{3.磁盘的访问时间Ta}}

磁盘的访问时间Ta表示为:\textbf{访问时间=寻道时间+旋转延迟+传输时间。}

\textbf{a.
寻道时间Ts:}磁盘接收到读指令后,磁头从当前位置移动到目标磁道位置,所需时间为寻道时间Ts。该时间是启动磁臂的时间s与磁头移动n条磁道所花费时间的总和,m为每移动一个磁道所需时间,即\textbf{{Ts=m×n+s};}

\textbf{b.
旋转延迟Tr:}旋转磁盘、定位数据所在的扇区所需的时间为旋转延迟Tr。设磁盘的旋转速度为r,则\textbf{{Tr=(1/r)/2=1/(2r)};}

\textbf{c.
传输时间Tt:}从磁盘上读取数据的时间为传输时间Tt。传输时间取决于每次读写的字节数b和磁盘的旋转速度,即\textbf{{Tt=b/(rN)},}式中,r为转速;N为一个磁道上的字节数。
