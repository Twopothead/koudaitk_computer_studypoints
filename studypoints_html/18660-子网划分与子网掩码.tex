\textbf{{1. 子网划分}}

聪明的人类想出了``{\textbf{子网号字段}}'',使得两级的IP地址变为三级的IP地址,这种做法叫做子网划分。子网划分属于一个单位内部的事情,单位对外仍然表现为没有划分子网的网络。

\textbf{划分子网的基本思路:}从主机号借用若干个比特作为子网号,而主机号也就相应减少了若干个比特,网络号不变。于是三级的IP地址可记为

~ ~ ~ ~ ~ ~\textbf{IP地址::={}

凡是从其他网络发送给本单位某个主机的IP分组,仍然根据IP分组的目的网络号先找到连接在本单位网络上的路由器,然后此路由器在收到IP分组后,再按目的网络号和子网号找到目的子网。最后将该IP分组直接交付给目的主机。

\textbf{{2. 子网掩码}}

子网划分与否是看不出来的,如果要告诉主机或路由器是否对一个A类、B类、C类网络进行了子网划分,则需要{\textbf{子网掩码}}。

子网掩码是一个与IP地址相对应的32位的二进制串,它由一串1和0组成。其中,1对应于IP地址中的网络号和子网号,0对应于主机号。因为1对1进行与操作,结果为1;1对0进行与操作,结果为0。所以使用一串1对网络号和子网号进行与操作,就可以得到网络号。

{现在的因特网标准规定,所有网络都必须有一个{\textbf{子网掩码}}。如果一个网络没有划分子网,就采用默认子网掩码。A类、B类、C类地址的默认子网掩码分别是255.0.0.0、255.255.0.0和255.255.255.0。}

\textbf{{总结:}{不管网络有没有划分子网,只要将子网掩码和IP地址进行逐位的``与''运算,就一定能立即得出网络地址。}}
