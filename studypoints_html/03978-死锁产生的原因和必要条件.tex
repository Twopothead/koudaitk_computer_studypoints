\textbf{{1.资源分类}}

\textbf{可剥夺资源:}是指虽然资源占有者进程需要使用该资源,但另一个进程可以强行把该资源从占有者进程处剥夺来归自己使用。\\
\textbf{不可剥夺资源:}是指除占有者进程不再需要使用该资源而主动释放资源,其他进程不得在占有者进程使用资源过程中强行剥夺。

\textbf{{2.死锁产生的原因}}

\textbf{a.} 死锁产生的原因是{\textbf{竞争资源}};\\
\textbf{b.}
虽然资源竞争可能导致死锁,但是\textbf{资源竞争并不等于死锁},{\textbf{死锁产生的原因是系统资源不足和进程推进顺序不当}};\\
\textbf{c.} {\textbf{系统资源不足}}是产生死锁的根本原因。

\textbf{{3.死锁产生的必要条件}}\\
\textbf{a.} 互斥条件;\\
\textbf{b.} 不剥夺条件;\\
\textbf{c.} 请求与保持条件;\\
\textbf{d.} 环路等待条件;\\
要产生死锁,这4个条件缺一不可,因此可以\textbf{通过破坏其中的一个或几个条件来避免死锁}的产生。
