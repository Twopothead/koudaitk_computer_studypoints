队列应用的例子,其实就是需要``排队''的情形,{举例如下。}

\textbf{{1. CPU资源的竞争问题}}

{在具有多个终端的计算机系统中,有多个用户需要使用CPU各自运行自己的程}{序,它们分别通过各自终端向操作系统提出使用CPU的请求,操作系统按照每个}{请求在时间上的先后顺序,将其排成一个队列,\textbf{{每次把CPU分配给队头用户使}}}{\textbf{用,当相应的程序运行结束,则令其出队,再把CPU分配给新的队头用户,直到}}{\textbf{{所有用户任务处理完毕}}{。}}

\textbf{{2. 主机与外部设备之间速度不匹配的问题}}

以主机和打印机为例来说明(\textbf{{09年真题已经考查}}),主机输出数据给打印机打{印,主机输出数据的速度}{比打印机打印的速度要快得多,若直接把输出的数据送}{给打印机打印,由于速度}{不匹配,显然是不行的。所以解决的方法是设置一个打}{印数据缓冲区,主机把要}{打印输出的数据依次写入到这个缓冲区中,写满后就暂}{停输出,继而去做其他的}{事情,打印机就从缓冲区中按照先进先出的原则依次取}{出数据并打印,打印完后}{再向主机发出请求,主机接到请求后再向缓冲区写入打}{印数据,这样利用队列既}{保证了打印数据的正确,又使主机提高了效率。}
