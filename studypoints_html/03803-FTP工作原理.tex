\textbf{{1. FTP协议的基本介绍}}

{{a.
基本概念:}文件传送协议(FTP)是因特网上使用的最广泛的传送协议。}FTP提供交互式的访问,允许客户指明文件的类型与格式,并允许文件具有存取权限。FTP屏蔽了各计算机系统的细节,因而\textbf{适合于在异构网络中任意计算机之间传送文件。}

b.
提供的服务:FTP\textbf{只提供文件传送的一些基本服务},它使用TCP可靠地传输服务。FTP使用客户/服务器模型。一个FTP服务器进程可同时为多个客户进程提供服务。

\textbf{{2. FTP协议的工作原理}}

FTP的服务器进程由两大部分组成:一个主进程负责接收新的请求;另外有若干个从属进程,负责处理单个请求。主进程的工作步骤如下:

1)打开熟知端口(端口号为21),使客户进程能够连接上。

2)等待客户进程发出连接请求。

3)启动从属进程来处理客户进程发来的请求。从属进程对客户进程的请求处理完毕后即终止,但从属进程在运行期间根据需要还可能创建一些其他子进程。

4)回到等待状态,继续接收其他客户进程发来的请求。主进程与从属进程的处理是并发进行的。
