由于Cache的内容只是主存部分内容的副本,因此它应当与主存内容保持一致。而CPU对Cache的写入更改了Cache的内容,就会导致Cache的内容和主存的内容不一致。如何能让Cache的内容与主存的内容保持一致就是Cache写操作策略需要完成的事情。\textbf{{Cache写操作策略有如下3种形式。}}

\textbf{1.写回法}

\textbf{写回法要求:}当CPU写Cache命中时,只修改Cache的内容,而不立即写入主存,只有当此行被换出时才写回主存,这种方式可以减少访问主存的次数。问题来了,那当换出此块的时候怎么能知道此块被修改过?实现这种方式时对Cache的每行都必须设置一个修改位(或者称为``脏位'')。当某行被换出时,根据此行的修改位是0还是1(可以规定1代表修改过,0代表没有修改),来决定将该行内容写回主存还是简单弃去。

{注意:上面考虑的是Cache命中时,那不命中呢?}如果CPU要对Cache中某块的某字进行修改,此时恰好此字不在Cache中,就需要从主存中找出包含此字的数据块。千万注意,CPU不会在主存中直接修改,而是找到之后直接复制到Cache中进行修改,等从Cache中换出此块时,再复制到主存。

\textbf{{此知识点可设置综合题的细节题,}}如当使用写回法时,求Cache的位数。此时,一些考生可能不会加上修改位(隐含条件,有多少行就加多少修改位)。

\textbf{2.全写法}

\textbf{全写法要求:}当写Cache命中时,Cache与主存同时发生写修改,因而较好地保持了Cache与主存内容的一致性。很明显,此时Cache不需要每行都设置修改位。当写Cache未命中时,直接在主存中修改(和写回法不同)。至于在主存中修改后需不需要复制到Cache中,这个视情况而定,可以复制也可以不复制。

\textbf{3.写一次法}

写一次法是基于写回法并结合全写法的写策略的一种形式(这种情况好像看得比较多,每次都是先介绍两种方式,第3种就采取折中方式,如Cache的映射方式就是如此)。写命中与写未命中的处理方法与写回法基本相同,仅仅是第一次写命中时要同时写入主存。
