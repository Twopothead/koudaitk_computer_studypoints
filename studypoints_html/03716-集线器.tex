\textbf{{1. 集线器的基本概念}}

中继器是普通集线器的前身,\textbf{{集线器实际就是一种多端口的中继器}}。{集线器一般有4、8、16、24、32等数量的接口,通过这些接口,集线器便能为相应数量的计算机完成``中继''功能。由于它在网络中处于一种``中心''位置,因此集线器也叫做Hub。}

{\textbf{2. 集线器的工作原理}}

假设有一个8个接口的集线器,共连接了8台计算机。集线器处于网络的``中心'',通过集线器对信号进行转发,可以实现8台计算机之间的互连互通。

\textbf{集线器通信过程模拟:}

假如计算机1要将一条信息发送给计算机8,当计算机1的网卡将信息通过双绞线送到集线器上时,集线器并不会直接将信息送给计算机8,它会将信息进行``广播'',即将信息同时发送给其他7个端口。当其他7个端口上的计算机接收到这条广播信息时,会对信息进行检查,如果发现该信息是发给自己的,则接收,否则不予理睬。由于该信息是计算机1发给计算机8的,因此最终计算机8会接收该信息,而其他6台计算机检查信息后,会因为信息不是发给自己的而不接收该信息。
