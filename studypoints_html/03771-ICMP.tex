ICMP报文分为两种,\textbf{即ICMP差错报告报文和ICMP询问报文。}

\textbf{{1. ICMP差错报告报文的分类}}

a. 终点不可达:\textbf{当路由器或主机不能交付数据报时发};

b. 源站抑制:\textbf{当路由器或主机由于拥塞而丢弃数据报时发};

c. 时间超过:\textbf{当IP分组的TTL值被减为0后发};

d.
参数问题:\textbf{当路由器或目的主机收到的数据报的首部中有字段的值不正确时发};

e. 改变路由(重定向):\textbf{有更好路由的时候发}。

\textbf{{2. ICMP询问报文的分类}}

1. 有回送请求和回答报文;

2. 时间戳请求和回答报文;

3. 掩码地址请求和回答报文;

4. 路由器询问和通告报文。

\textbf{{3. 不应发送ICMP差错报告报文的几种情况}}

1. 对\textbf{ICMP差错报告报文}不再发送ICMP差错报告报文;

2.
对\textbf{第一个分片的数据报片的所有后续数据报片}都不发送ICMP差错报告报文;

3. 对\textbf{具有组播地址的数据报}都不发送ICMP差错报告报文;

4.
对\textbf{具有特殊地址}(如127.0.0.0或0.0.0.0)的数据报不发送ICMP差错报告报文。
