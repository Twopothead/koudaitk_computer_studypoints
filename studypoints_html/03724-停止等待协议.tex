从名称上来看,也可以看出停止-等待协议是基于{停止-等待流量控制技术}的。从滑动窗口的角度来理解就是其发送窗口大小为1,接收窗口大小也为1。

\textbf{停止-等待协议的基本思想:}发送方传输一个帧后,必须等待对方的确认才能发送下一帧。若在规定时间内没有收到确认,则发送方超时,并重传原始帧。看到这里也许有人会问,停止-等待流量控制技术(这里是停止-等待流量控制技术而不是停止-等待协议)为什么要一直等待?为什么不设置一个规定时间?这里就要回到第1章协议的制定。首先协议需要建立在一定技术(停止-等待流量控制技术)之上,然后在此技术之上需要考虑一切可能突发的不利状况(\textbf{可以这么理解:}协议=技术+考虑不利因素,即停止-等待协议=停止-等待流量控制技术+不利因素),设置规定时间重传就是为了解决这些不利因素。如果不设置时间就会造成死锁,这样就无法推进,在这里可以联系到操作系统的死锁,如果没有外力参与去打破死锁,就会一直等待下去,而这里的外力就是重传计时器。

\textbf{{停止-等待协议中会出的差错主要有以下两类}}(虽然简单,请仔细看,这里有很多考生的疑问点,其他辅导书都没有涉及)。

\textbf{1)帧一般被分为数据帧和确认帧。}第一类错误就是数据帧被损坏或者丢失,那么接收方在进行差错检验时,会检测出来。处理数据帧被损坏的情况时,使用\textbf{计时器}即可解决。这样发送方在发送一个帧后,若数据能够正确地接收到,接收方就发送一个确认帧,没有问题;若接收方收到的是一个被损坏的数据帧,则直接丢弃,此时发送方还在那里苦等,不过没有关系,{只要计时器超时了,发送方就会重新发送该数据帧},如此重复,直到这一数据帧无错误地到达接收方为止。

\textbf{2)第二类错误是确认帧被破坏或者丢失。}一旦确认帧被破坏或者丢失,造成的后果就是发送方会不断地重新发送该帧,从而导致接收方不断地重新接收该帧。怎么解决?显然,对于接收方而言,需要有能够区分某一帧是新帧还是重复帧的能力。解决方法很简单,就是让发送方在每个待发的帧的头部加一个编号,而接收方对每个到达的帧的编号进行识别,{判断是新帧还是要抛弃的重复帧}。
