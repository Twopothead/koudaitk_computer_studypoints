\textbf{{1.I/O设备的分类}}

{a. 按设备的使用特性分类:存储设备和I/O设备;}

b. 按信息交换单位分类:字符设备和块设备;

c. 按传输速率分类:低速设备、中速设备和高速设备;

d. 按设备的共享属性分类:独占设备、共享设备和虚拟设备。

\textbf{{2.I/O管理的任务和功能}}

{a.
设备分配:}{按照设备类型和相应的分配算法决定将I/O设备分配给哪一个进程。}

{b. 设备处理:}{设备处理程序用以实现CPU和设备控制器之间的通信。}

{c.
缓冲管理:}{设置缓冲区的目的是为了缓和CPU与I/O设备速度不匹配的矛盾。}

{d.
设备独立性:}{设备独立性又称为设备无关性,是指应用程序独立于物理设备。}
