{{\textbf{1. 文件的实现}}}

{{a.
基本概念:文件的实现\textbf{主要是指文件在存储器上的实现,即文件物理结构的实现,包括外存分配方式与文件存储空间的管理}。}{}{}}

\textbf{{2.外存分配方式(连续分配)}}

\textbf{a.
基本概念:}{把逻辑文件中的记录顺序地存储到相邻的物理盘块中,这样所形成的文件结构称为顺序文件结构,此时的物理文件称为顺序文件。}

\textbf{b.
优点:}查找速度比其他方法快(只需要起始块号和文件大小),目录中关于文件物理存储位置的信息也比较简单。

\textbf{c.
缺点:}容易产生碎片,需要定期进行存储空间的紧缩。{这种分配方法不适合文件随时间动态增长和减少的情况,也不适合用户事先不知道文件大小的情况。}

\textbf{{{3}\textbf{{.外存分配方式(链接分配)}}}}

\textbf{a. 隐式链接:}{用于链接物理块的指针隐式地放在每个物理块中。}

\textbf{b.
显式链接:}用于链接物理块的指针显式存放在内存的一张链接表中。\\

\textbf{c. 优点:}简单(只需起始位置),文件创建与增长容易实现。

\textbf{d.
缺点:}不能随机访问盘块,链接指针会占用一些存储空间,而且存在可靠性问题。

\textbf{\textbf{{\textbf{4.外存分配方式(索引分配)}}}}

\textbf{a.
基本概念:}系统为每个文件分配一个索引块,索引块中存放索引表,索引表中的每个表项对应分配给该文件的一个物理块。

\textbf{b.
单级索引分配:}单级索引分配方法就是将每个文件所对应的盘块号集中放在一起,为每个文件分配一个索引块(表),再把分配给该文件的所有盘块号都记录在该索引块中,因而该索引块就是一个包含多个盘块号的数组。

\textbf{c.
两级索引分配:}当文件较大,一个索引块放不下文件的块序列时,可以对索引块再建立索引,这样构成二级索引。

\textbf{d.
混合索引分配:}{所谓混合索引分配,是指将多种索引分配方式相结合而形成的一种分配方式。}

\textbf{{{5.文件存储空间管理(}\textbf{{空闲文件表}}{)}}}

\textbf{a.
基本概念:}{为所有空闲文件单独建立一个目录,每个空闲文件在这个目录中占一个表目。表目的内容包括第一个空闲块号、物理块号和空闲块数目。}

\textbf{b.
优点:}{当文件存储空间中只有少量空闲文件时,这种方法有较好的效果。}

\textbf{c.
缺点:}如果存储空间中有大量的小空闲文件,则空闲文件目录将变得很大,其效率将大为降低。这种管理技术仅适用于连续文件。

\textbf{\textbf{{{6.文件存储空间管理(}\textbf{{空闲块链表}}{)}}}}

\textbf{a.
基本概念:}将文件存储设备上的所有空闲块链接在一起,形成一条空闲块链,并设置一个头指针指向空闲块链的第一个物理块。也可以将链表中的空闲盘块改为空闲盘区(每个空闲盘区包含若干个连续的空闲盘块),这样的链称为空闲盘区链。

\textbf{\textbf{{7.文件存储空间管理(位示图法)}}}

\textbf{a.
基本概念:}为文件存储器\textbf{建立一张位示图(称它是图,其实就是一连串的二进制位)},以反映整个存储空间的分配情况。在位示图中,每一个二进制位都对应一个物理块,若某位为1,表示对应的物理块已分配;若为0,表示对应的物理块空闲。
