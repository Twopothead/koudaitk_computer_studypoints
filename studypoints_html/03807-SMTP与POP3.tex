\textbf{{1.SMTP}}

a.
基本介绍:\textbf{简单邮件传送协议(SMTP)}所规定的就是在两个相互通信的SMTP进程之间应如何交换信息。SMTP运行在TCP基础之上,{\textbf{使用25号端口}},也\textbf{{使用客户/服务器模型}}。

\includegraphics{file://C:/Users/ADMINI~1/AppData/Local/Temp/SGTpbq/4656/03F8434B.gif}b.
通信过程:

\textbf{1)连接建立。}连接是在发送主机的SMTP客户和接收主机的SMTP服务器之间建立的。SMTP不使用中间的邮件服务器。

\textbf{2)邮件传送。}

\textbf{3)连接释放:}邮件发送完毕后,SMTP应释放TCP连接。

\textbf{{2.POP3}}

邮局协议(POP)是一个非常简单,但功能有限的邮件读取协议。现在使用的是它的第三个版本POP3。POP\textbf{{也使用客户/服务器的工作方式}}。在接收邮件的用户计算机中必须运行POP客户程序,而在用户所连接的ISP的邮件服务器中运行POP服务器程序。

\textbf{{注意:}只要用户从POP服务器读取了邮件,POP服务器就将该邮件删除。}
