{\textbf{{虚拟存储器的相关概念归纳如下:}} }

{\textbf{{1)}}{虚拟存储器是一个逻辑模型,}并不是一个实际的物理存储器。}

{{\textbf{2)}虚拟存储器必须建立在主存-辅存结构基础上,}但两者是有差别的:虚拟存储器允许使用比主存容量大得多的地址空间,并不是虚拟存储器最多只允许使用主存空间;虚拟存储器每次访问时,必须进行虚实地址变换,而非虚拟存储器则不必。}

{\textbf{3)}虚拟存储器的作用是分隔地址空间,解决主存的容量问题和实现程序的重定位。}

{\textbf{4)}虚拟存储器和Cache都基于程序局部性原理。}

{\textbf{两者的相同点:}都把程序中最近常用的部分驻留在高速的存储器中;一旦这部分程序不再常用,把它们送回到低速存储器中;这种换入、换出操作是由硬件或操作系统完成的,对用户透明;都力图使存储系统的性能接近高速存储器,而价格却接近低速存储器。}

{\textbf{两者的不同点:Cache用硬件实现,对操作系统透明,}而虚拟存储器用操作系统与硬件相结合的方式实现;Cache是一个物理存储器,而虚拟存储器仅是一个逻辑存储器,其物理结构建立在主存-辅存结构基础上。其实\textbf{,}此知识点中更需要掌握的是为了实现这种虚拟存储管理而需要的技术手段,如请求分页存储管理、请求分段存储管理、请求段页式存储管理,但这些知识点都属于操作系统的内容,可以参考操作系统对于此块知识点的讲解。}
