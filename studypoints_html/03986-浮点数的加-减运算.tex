对于浮点数的加/减运算,可以\textbf{{总结为以下4个步骤:}}

\textbf{① 对阶,}使两数的小数点位置对齐。

\textbf{② 尾数求和,}将对阶后的两尾数按定点加/减运算规则求和或者求差。

\textbf{③
规格化,}为增加有效数字的位数,提高运算精度,必须将求和或求差后的尾数规格化。

\textbf{④ 舍入,}为提高精度,要考虑尾数右移时丢失的数值位。

当然,以上4个步骤完成后,还需要加上一步,即\textbf{{检查一下最后的结果是否溢出,}}由于浮点数的溢出完全是用阶码来判断的,假设阶码采用补码来表示,溢出就可以使用双符号位判断溢出的方式来判断此浮点数是否溢出,过程如下:

~ ~ ~ ~ ~ if(阶符= =01)\\
\hspace*{0.333em} ~ ~ ~ ~ ~ ~ 上溢,需做中断处理;\\
\hspace*{0.333em} ~ ~ ~ ~ else ~if(阶符= =10)\\
\hspace*{0.333em} ~ ~ ~ ~ ~ ~ 下溢,按机器零处理;\\
\hspace*{0.333em} ~ ~ ~ ~ else\\
\hspace*{0.333em} ~ ~ ~ ~ ~ ~ 结果正确;\\
