\textbf{\textbf{{1.}}{}{BGP的基本原理}}

每一个自治系统的管理员要选择至少一个路由器(可以有多个)作为该自治系统的``\textbf{BGP发言人}''。一个BGP发言人要与其他自治系统中的BGP发言人交换路由信息,\textbf{{就要先建立TCP连接}},然后在此连接上交换BGP报文以建立BGP会话,再利用BGP会话交换路由信息。各BGP发言人互相交换网络可达性的信息后,各BGP发言人就可找出到达各自治系统比较好的路由了。

\textbf{{2.BGP的特点}}

\textbf{1)}BGP交换路由信息的结点数量级是自治系统数的量级,这要比自治系统中的网络数少很多。

\textbf{2)}每一个自治系统中BGP发言人(或边界路由器)的数目是很少的,这样就使得自治系统之间的路由选择不过分复杂。

\textbf{3)}BGP支持
CIDR,因此BGP的路由表也就应当包括目的网络前缀、下一跳路由器以及到达该目的网络所要经过的各个自治系统序列。

\textbf{4)}在BGP刚刚运行时,BGP的邻站是交换整个的BGP路由表,但以后只需要在发生变化时更新有变化的部分。这样对节省网络带宽和减少路由器的处理开销方面都有好处。

\textbf{{3.BGP的4种报文}}

\textbf{1)}打开报文。用来与相邻的另一个BGP发言人建立关系。

\textbf{2)}更新报文。用来发送某一路由的信息以及列出要撤销的多条路由。

\textbf{3)}保活报文。用来确认打开报文和周期性地证实邻站关系。

\textbf{4)}通知报文。用来发送检测到的差错。\\
