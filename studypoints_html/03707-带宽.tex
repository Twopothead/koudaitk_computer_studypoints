\textbf{带宽}分为\textbf{模拟信号的带宽和数字信号的带宽}。

在过去很长一段时间里,通信的主干线路传送的是模拟信号,此时带宽的定义为:通信线路允许通过的信号频带范围,就是允许通过的最高频率减去最低频率,例如某通信线路允许通过的最低频率为300Hz,最高频率为3400Hz,则该通信线路的带宽就为3100Hz。

但是,{在计算机网络中,带宽不是以上的定义。此时的带宽是用来表示网络的通信线路所能传送数据的能力。因此,带宽表示在单位时间内从网络中的某一点到另一点所能通过的``最高数据率''。显然,}{此时带宽的单位不再是Hz,而是bit/s,读作``比特每秒''。}
