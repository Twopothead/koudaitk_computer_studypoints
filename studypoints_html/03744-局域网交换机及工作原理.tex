\textbf{{1.局域网交换机的基本概念}}

\textbf{局域网交换机实质上是多端口网桥,它工作在数据链路层。}局域网交换机的每个端口都直接与主机或集线器相连,并且一般都工作在全双工方式。当主机需要通信时,交换机能同时连通许多对的端口,使每一对相互通信的主机都能像独占通信媒体那样,进行无冲突的传输数据,通信完成后断开连接。

\textbf{{2. 交换机总容量计算方式}}

a. 如果是半双工:总容量=端口数×每个端口带宽;

b. 如果是全双工:总容量=端口数×每个端口带宽×2。

\textbf{{3. 交换机的两种交换模式}}

a.
直通式交换:\textbf{只检查帧的目的地址},这使得帧在接收后能马上被转发出去。这种方式速度很快,但缺乏安全性,也无法支持具有不同速率的端口的交换。

b.
存储转发式交换:先将接收到的帧存储在高速缓存中,并检查数据是否正确,确认无误后,查找转发表,并将该帧从查询到的端口转发出去。\textbf{{如果发现该帧有错误,就将其丢弃}}。存储转发式交换的优点是可靠性高,并能支持不同速率端口间的转换,其缺点是延迟较大。

\textbf{{4.局域网交换机的工作原理}}

与网桥类似,检测从某端口进入交换机的帧的源MAC地址和目的MAC地址,然后与系统内部的动态查找表进行比较,{\textbf{若数据报的MAC地址不在查找表中,则将该地址加入查找表中,并将数据报发送给相应的目的端口}}。
