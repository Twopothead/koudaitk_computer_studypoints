\textbf{{1.进程的定义}}

\textbf{a.} 进程是程序在处理器上的一次执行过程;\\
\textbf{b.} 进程是可以和别的进程并行执行的计算;\\
\textbf{c.}
进程是程序在一个数据集合上的运行过程,是系统进行资源分配和调度的一个独立单位;\\
\textbf{d.} 进程可定义为一个数据结构及能在其上进行操作的一个程序;\\
\textbf{e.}
进程是一个程序关于某个数据集合在处理器上顺序执行所发生的活动。

\textbf{{2.进程的特征}}

进程具有以下几个基本特征:\textbf{{动态性、}{并发性、}{独立性、}{异步性}。}

\textbf{进程的结构特征:}为了描述和记录进程的运动变化过程,并使之能正确运行,应为每个进程配置一个进程控制块(Process
Control
Block,PCB)。这样从结构上看,每个进程都由\textbf{程序段、数据段}和一个\textbf{进程控制块}组成。

\textbf{{3.进程和程序的关系}}

\textbf{a.~}进程是动态的,程序是静止的;

\textbf{b.~}进程是暂时的,程序是永久的;\\

\textbf{c.~}进程与程序的组成不同:进程的组成包括程序、数据和进程控制块;

\textbf{d.~}通过多次执行,一个程序可以产生多个不同的进程;通过调用关系,一个进程可以执行多个程序。进程可创建其他进程,而程序不能形成新的程序;\\

\textbf{e.~}进程具有并行特性(独立性、异步性),程序则没有。

\textbf{{{4.进程和作业的区别}\\
}}

\textbf{a.}
{作业是用户向计算机提交任务的任务实体。而进程则是完成用户任务的执行实体,是向系统申请分配资源的基本单位;}

\textbf{b.}
一个作业可由多个进程组成,且必须至少由一个进程组成,但一个进程不能构成多个作业。\\
\textbf{c.}
作业的概念主要用在批处理系统中。像UNIX这样的分时系统则没有作业的概念;而进程的概念则用在几乎所有的多道程序系统中。\\

\textbf{{5.进程的组成}}

进程由进程控制块(PCB)、程序段、数据段、进程标识符(PID)、进程当前状态、进程队列指针、程序和数据地址、进程优先级、CPU现场保护区、通信信息、家族联系、占有资源清单组成。
