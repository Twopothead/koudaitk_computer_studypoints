{步骤一:确定题目涉及的若干进程中\textbf{哪些进程是并发的}}{{,}{分析清楚\textbf{若干进程间的制约关系}}}{(互斥或同步);}

{步骤二:根据刚才对制约关系和并发关系的分析,确定进程流程}{,}{依据进程流程和进程间的制约关系\textbf{设置相关信号量或变量以及它们的初值}}{(这里要明确每种信号量和变量的物理含义,以及初值的含义);}

{步骤三:将设置好的信号量\textbf{添加到进程流程的适当位置;}}

{步骤四:根据添加好信号量的进程流程\textbf{写出伪代码;}}

{步骤五:\textbf{检查伪代码是否正确完整}。}\\

{{步骤六:(可选)最后设}{{}{{}{{}{}}}}{计一个简单的用例\textbf{测试一下伪代码的正确性}。}}
