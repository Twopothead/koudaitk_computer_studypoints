{\textbf{检错编码:通过一定的编码和解码,能够在接收端解码时检查出传输的错误,但不能纠正错误。常见的检错编码有奇偶校验码和循环冗余码(CRC)。}}

\textbf{1.奇偶校验码}

奇偶校验码就是在信息码后面加一位校验码,分奇校验和偶校验。

\textbf{奇校验:}添加一位校验码后,使得整个码字里面1的个数是奇数。接收端收到数据后就校验数据里1的个数,若检测到奇数个1,则认为传输没有出错;若检测到偶数个1,则说明传输过程中,数据发生了改变,要求重发。

\textbf{偶校验:}添加一位校验码后,使得整个码字里面1的个数是偶数。接收端收到数据后就校验数据里1的个数,若检测到偶数个1,则认为传输没有出错;若检测到奇数个1,则说明传输过程中,数据发生了改变,要求重发。

\textbf{2.循环冗余码}

奇偶校验码的检错率极低,不实用。目前,在计算机网络和数据通信中,用得最广泛的是检错率极高、开销小、易实现的循环冗余码(CRC)。循环冗余码的原理比较简单,这里就不再赘述了,教材讲解得很细致。
