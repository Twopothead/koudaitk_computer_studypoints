\textbf{{1. 假脱机技术}}

系统中独占设备的数量有限,往往不能满足系统中多个进程的需要,从而成为系统的``瓶颈'',使许多进程因等待而阻塞。另一方面,分配到独占设备的进程,在整个运行期间往往占有但不经常使用设备,使设备利用率偏低。为克服这种缺点,\textbf{人们通过共享设备来虚拟独占设备,将独占设备改造成共享设备,从而提高了设备利用率和系统的效率},该技术称为假脱机(SPOOLing)技术。

\textbf{{2.~}\textbf{{SPOOLing技术}}{}}

{\textbf{a. 基本概念}}

是\textbf{{低速输入输出设备与主机交换的一种技术}},其核心思想是\textbf{{以联机的方式得到脱机的效果}}。低速设备经通道和设在主机内存的缓冲存储器与高速设备相连,该高速设备通常是辅存。为了存放从低速设备上输入的信息,在内存中形成缓冲区,在高速设备上形成输出井和输入井,传递时信息从低速设备传入缓冲区,再传到高速设备的输入井,再从高速设备的输出井传到缓冲区,再传到低速设备。

\textbf{{b.SPOOLing系统的组成}}

1)输入井和输出井:输入井和输出井是在磁盘上开辟出来的两个存储区域。\textbf{输入井模拟脱机输入时的磁盘,用于收容I/O设备输入的数据。输出井模拟脱机输出时的磁盘,用于收容用户程序的输出数据。}

2)输入缓冲区和输出缓冲区:输入缓冲区和输出缓冲区是在内存中开辟的两个缓冲区。\textbf{输入缓冲区用于暂存由输入设备传递过来的数据,然后再传送到输入井。输出缓冲区用于暂存从输出井传递过来的数据,然后再传送到输出设备}。

3)输入进程和输出进程:\textbf{输入进程模拟脱机输入时的外围控制机},将用户要求的数据从输入设备通过输入缓冲区再传递到输出井。当需要输入数据时,CPU直接将数据从输入井读入内存。{\textbf{输出进程}}{\textbf{模拟}}{\textbf{脱机输出时的外围控制机},把用户要求输出的数据先从内存送到输出井,等输出设备空闲时,再将输出井中的数据经过输出缓冲区送到输出设备上。}

\textbf{{c.SPOOLing技术的特点}}

{1)提高了I/O速度;}

{2)设备并没有分配给任何进程;}

{3)实现了虚拟设备功能;}

{4)SPOOLing除了是一种速度匹配技术外,也是一种虚拟设备技术。}
