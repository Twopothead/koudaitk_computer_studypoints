\textbf{纠错编码:就是在接收端不但能检查错误,而且能纠正检查出来的错误。}

常见的纠错编码是海明码。

\textbf{海明码:}又称为汉明码,它是在信息字段中插入若干位数据,用于监督码字里的哪一位数据发生了变化,具有一位纠错能力。

假设信息位有k位,整个码字的长度就是k+r位;每一位的数据只有两种状态,不是1就是0,有r位数据就能表示出2\textsuperscript{r}种状态。如果每一种状态代表一个码元发生了错误,有k+r位码元,就要有k+r种状态来表示,另外还要有一种状态来表示数据正确的情况,所以{2}\textsuperscript{{r}}{-1≥k+r}才能检查一位错误,即2\textsuperscript{r}≥k+r+1。例如,信息数据有4位,由2\textsuperscript{r}≥k+r+1得r≥3,也就是至少需要3位监督数据才能发现并改正1位错误。

\textbf{海明码求解的具体步骤如下:}\\
1)确定校验码的位数r。\\

2)确定校验码的位置。

3)确定数据的位置。

4)求出校验位的值。

\textbf{{实战例题请见《计算机网络高分笔记》。}}

{考生在教材中肯定见到过以下公式:}

\textbf{L-1=D+C ~ ~且D≥C}

如果要纠正d位错误,说明至少要检测出d位错误(当然可以检测得更多),代入即可得到L-1=d+d,即L=2d+1。
