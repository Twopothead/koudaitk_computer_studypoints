{\textbf{背景知识:}}为什么要使用总线结构?\textbf{早期的计算机}大都采用分散连接方式,它是以存储器为中心的结构,由于现在的外部设备太多,如果使用分散连接方式,则随时增减外部设备不易实现,因此衍生出了\textbf{总线}的结构。

\textbf{1)总线的传输周期。}

指CPU通过总线对存储器或I/O端口进行一次访问所需的时间,包括\textbf{总线申请阶段、寻址阶段、传输阶段和结束阶段}(在总线操作和定时里将详细讲解这4个阶段)。

\textbf{2)总线宽度。}

总线实际上是由许多传输线或通路组成的,每条线可一位一位地传输二进制代码,一串二进制代码可在一段时间内逐一传输完成。若干条传输线可以同时传输若干位二进制代码,如16条传输线组成的总线可同时传输16位二进制代码。

\textbf{3)总线特性。}

这个不重要,记住每个功能特性大致做什么就行,总结如下。

\textbf{① 机械特性:}尺寸、形状。\\
\textbf{② 电气特性:}传输方向和有效的电平范围。\\
\textbf{③
功能特性:}每根传输线的功能(是传送地址?还是传送数据?或是传送控制信号?)。\\
\textbf{④ 时间特性:}信号的时序关系(哪根线在什么时间内有效)。
