\textbf{{1.手工操作阶段}}

使用过程大致如下:先将程序纸带(或卡片)装入输入机,然后启动输入机把程序和数据送入计算机,接着通过控制台开关启动程序运行,当程序运行完毕后,由用户取走纸带和结果。

由此可以推断出,这种操作方式具有{用户独占计算机资源、资源利用率低以及CPU等待等人工操作}的特点。

\textbf{{2.脱机输入输出技术}}

脱机输入输出技术是{为了解决CPU和I/O设备之间速度不匹配的矛盾}而提出的,此技术减少了CPU的空闲等待时间,提高了I/O速度。

采用脱机输入输出技术后,低速I/O设备上数据的输入输出都在外围机的控制下进行,而CPU只与高速的输入带及输出带打交道,从而{有效地减少了CPU等待慢速设备输入输出的时间}\textbf{。}

\textbf{{3.批处理技术}}

{批处理技术是指计算机系统对一批作业自动进行处理的一种技术}\textbf{{。}}计算机系统对磁带上的作业自动地一个接一个进行处理,直至把磁带上的所有作业全部处理完毕,这样便形成了早期的批处理系统。

\textbf{{4.多道程序设计技术}}

在早期批处理系统中,每次只将一个用户程序调入内存运行,这种作业运行方式称为单道运行。缺点是{每当程序发出I/O请求时,CPU便处于等待I/O完成的状态,致使CPU空闲。}{为进一步提高CPU的利用率,引入了多道程序设计技术。}

{多道程序设计技术是``将一个以上的作业存放在主存中,并且同时处于运行状态。这些作业共享处理器、外设以及其他资源''。现代计算机系统一般都基于多道程序设计技术。}

在单处理器系统中,多道程序运行的特点:{多道、}{宏观上并行、}{微观上串行}\textbf{。}

\textbf{{5.操作系统的形成}}

{操作系统}是一组{控制和管理}{计算机硬件和软件资源},{合理地组织}计算机工作流程,以及{方便用户}的程序的集合。
