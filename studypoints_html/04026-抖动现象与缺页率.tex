\textbf{{1.Belady异常}}

\textbf{a.
定义:}{FIFO置换算法的}\textbf{缺页率可能会随着所分配的物理块数的增加而增加,这种奇怪的现象就是Belady异常。}

\textbf{b.
原因:}FIFO算法的置换特征与进程访问内存的动态特征相矛盾,即被置换的页面并不是进程不会访问的。

\textbf{{注意:LRU算法和最佳置换算法永远不会出现Belady异常,被归类为堆栈算法的页面置换算法也不可能出现Belady异常。}}

\textbf{{2.抖动现象}}

\textbf{a.
定义:}{若选用的页面置换算法不合适,可能会出现抖动现象:刚被淘汰的页面,过后不久又要访问,并且调入不久后又调出,如此反复,使得}\textbf{系统把大部分时间用在了页面的调入调出上,而几乎不能完成任何有效的工作,这种现象称为抖动(或颠簸)。}

\textbf{b.
原因:}在请求分页系统中每个进程只能分配到所需全部内存空间的一部分。

\textbf{{3.缺页率}}

\textbf{a.
定义:}{假定一个作业共有n页,系统分配给该作业m页的空间(m≤n)。如果该作业在运行中共需要访问A次页面(即引用串长度为A),其中所要访问页面不在内存,需要将所需页调入内存的次数为F,则缺页率定义为f=F/A,命中率即为1-f。}

\includegraphics{file://C:/Users/ADMINI~1/AppData/Local/Temp/SGTpbq/4140/0F609C4F.gif}b.~缺页率会受\textbf{置换算法、分配的页面数量、页面大小}等因素的影响。

c.
缺页率对于请求分页管理系统是很重要的,如果\textbf{缺页率过高,会直接导致读取页面的平均时间增加,会使进程执行速度显著降低}。因此,如何降低缺页率是一项非常重要的工作。
