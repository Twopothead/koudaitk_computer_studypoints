\textbf{{计算机网络为什么要采用分层结构?}}\\
这里用一个小的生活实例来解释。任何一个公司都是从小企业创办而来的,当公司规模很小(比如只有一个老总和3个员工)时,老总和员工可以同处于一个平面,不需要分层,员工可以直接向老总汇报问题。但是,如果该公司是诸如微软这样的公司(也就是计算机网络具有相当大的规模时),比尔·盖茨当然处于最高层,他的作用就是实现公司的长远发展,而不可能每天与公司的员工讨论某功能模块应该使用哪种算法。同理,当网络结构大时,就必须要分层,并且每一层都需实现所对应的功能,这样才会有更好的发展。但是,分层又不能太多,如果分层太多,资源浪费就很多。所以,TCP/IP折中地采用了4层结构模型(在教材中为了更好地描述各层的工作原理经常被看作5层)。

{\textbf{3个专业术语}}

1.
\textbf{实体:}任何可发送或接收信息的硬件或软件进程,通常是一个特定的软件模块。\\
2. \textbf{对等层:}不同机器上的同一层。\\
3. \textbf{对等实体:}同一层上的实体。

\textbf{理解方式:}{A省和B省分别表示不同的机器,可将A省和B省的各层干部看成实体,将A省省长职位和B省省长职位看成对等层,而将此对等层上的实体,即A省省长和B省省长,可看成对等实体。}
