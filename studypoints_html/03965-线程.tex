{\textbf{{1. 线程的引入}}\\
}

进程的两个基本属性:\\
a. 进程是一个拥有资源的独立单元;\\
b. 进程同时又是一个可以被处理器独立调度和分配的单元。\\
为了使多个程序更好地并发执行,并尽量减少操作系统的开销,{操作系统设计者引入了线程,让线程去完成第二个基本属性的任务},而进程只完成第一个基本属性的任务。

{2.~}\textbf{{线程的定义}}

{线程是进程内一个相对独立的、可调度的执行单元。线程自己基本上不拥有资源,只拥有一点在运行时必不可少的资源(如程序计数器、一组寄存器和栈),但它可以与同属一个进程的其他线程共享进程拥有的全部资源。}{多线程是指一个进程中有多个线程,这些线程共享该进程资源。}\textbf{但是各线程自己堆栈数据不对其他线程共享}{。}

\textbf{{3. 线程的实现}}

{\textbf{a.
内核级线程:}}{指依赖于内核,由操作系统内核完成创建和撤销工作的线程}{。}

{{\textbf{b.
用户级线程}}\textbf{:}}{指不依赖于操作系统核心,由应用进程利用线程库提供创建、同步、调度和管理线程的函数来控制的线程}{。}

\textbf{{\textbf{{4.~}}{线程与进程的比较}}}

\textbf{{a.
调度:}}在传统的操作系统中,拥有资源和独立调度的基本单位都是进程。而在引入线程的操作系统中,线程是独立调度的基本单位,进程是拥有资源的基本单位。{在同一进程中,线程的切换不会引起进程切换。在不同进程中进行线程切换,如从一个进程内的线程切换到另一个进程中的线程中,将会引起进程切换。}

\textbf{{b.
拥有资源:}}进程是拥有资源的基本单位,而线程不拥有系统资源(也有一点必不可少的资源,并非什么资源都没有),但线程可以访问其隶属进程的系统资源。\\

\textbf{{c.
并发性:}}在引入线程的操作系统中,不仅进程之间可以并发执行,而且同一进程内的多个线程之间也可以并发执行。

\textbf{{d.
系统开销:}}由于创建进程或撤销进程时,系统都要为之分配或回收资源,系统开销较大;而线程切换时,只需保存和设置少量寄存器内容,因此开销很小。

\textbf{{5.多线程模型}}

\textbf{{a.
多对一模型:}}多对一模型将多个用户级线程映射到一个内核级线程上。\textbf{只要一个用户级线程阻塞,就会导致整个进程阻塞。}

\textbf{{b.
一对一模型:}}一对一模型将内核级线程与用户级线程一一对应。\textbf{这样做的好处是当一个线程阻塞时,不影响其他线程的运行}。

\textbf{{c.
多对多模型:}}多对多模型将多个用户级线程映射到多个内核级线程,采用这样的模型可以打破前两种模型对用户级线程的限制,不仅可以使\textbf{多个用户级线程真正意义上并行执行,而且不会限制用户级线程的数量。}
