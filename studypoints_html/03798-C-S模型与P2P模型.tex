\textbf{{1. 客户/服务器模型(C/S模型)}}

\textbf{客户(Client)和服务器(Server)都是指通信中所涉及的两个应用进程。}客户/服务器模型所描述的是进程之间的服务和被服务的关系。在这个模型中,\textbf{客户是服务的请求方,服务器是服务的提供方}。

\textbf{{客户/服务器模型主要特点:}}

\textbf{1)}网络中各计算机的地位不平等,服务器可以通过对用户权限的限制来达到管理客户机的目的,使它们不能随意存储数据,更不能随意删除数据,或进行其他受限的网络活动;

\textbf{2)}整个网络的管理工作由少数服务器承担;

\textbf{3)可扩展性不佳},由于受服务器硬件和网络带宽的限制,服务器所能支持的客户数比较有限,当客户数增长较快时,会急剧影响网络应用系统的效率。

\textbf{{2. P2P模型}}

\textbf{实际上,P2P模型从本质上来看仍然是使用客户/服务器方式,只是对等连接中的每一个主机既是客户又是服务器。}

P2P模型带来的好处是,任何一台主机都可以成为服务器,改变了原来需要专用服务器的模式,很显然,多个客户机之间可以直接共享文档。此外,可以借助P2P网络模型,解决专用服务器的性能瓶颈问题。

\textbf{{P2P模型主要特点如下:}}

\textbf{1)繁重的计算机任务可以被分配到各个结点上,}利用每个结点空闲的计算能力和存储空间,聚合实现强大的服务。

\textbf{2)系统可扩展性好。}传统的服务器有连接带宽的限制,只能达到一定的客户端连接数。但是在P2P模型中,能避免这个问题。

\textbf{3)网络更加健壮,不存在中心结点失效的问题。}当一部分结点连接失败之后,其余的结点仍然能形成完整的网络。
