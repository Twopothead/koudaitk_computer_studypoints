\textbf{{后退N帧协议}}基于滑动窗口流量控制技术。若采用n个比特对帧进行编号,其发送窗口尺寸WT必须满足1\textless{}W\textsubscript{T}\textbf{≤}2\textsuperscript{n}-1(请参考下面的补充知识点),接收窗口尺寸为1。若发送窗口尺寸大于{2}\textsuperscript{n}{-1},会造成接收方无法分辨新、旧数据帧的问题。由于接收窗口尺寸为1,因此接收方只能按序来接收数据帧。\\

\textbf{后退N帧协议的基本原理:}发送方发送完一个数据帧后,不是停下来等待确认帧,而是可以连续再发送若干个数据帧。如果这时收到了接收方的确认帧,那么还可以接着发送数据帧。\textbf{如果某个帧出错了,接收方只能简单地丢弃该帧及其所有的后续帧。}发送方超时后需重发该出错帧及其后续的所有帧。由于减少了等待时间,后退N帧协议使得整个通信的吞吐量得到提高。但接收方一发现错误帧,就不再接收后续的帧,造成了一定的浪费。据此改进,得到了选择重传协议。

\textbf{{补充知识点:为什么后退N帧协议的发送窗口尺寸WT必须满足1\textless{}WT≤}{2}\textsuperscript{{n}}{-1}{?}}

解析:假设发送窗口的大小为\textbf{2\textsuperscript{n}},发送方发送了0号帧,接收窗口发送ACK1(0号帧已收到,希望接收1号帧,但是ACK1丢失),接着发送方发送了1号帧,接收窗口发送ACK2(1号帧已收到,希望接收2号帧,但是ACK2丢失),以此类推,直到发送方发了第\textbf{2\textsuperscript{n}}-1号帧,接收方发送ACK\textbf{2\textsuperscript{n}}(丢失),此时不能再发送数据了,因为已经发送了\textbf{2\textsuperscript{n}}个帧,但一个确认都没有收到,所以过一段时间0号帧的计时器会到达预定时间进行重发,此时发过去接收方认为是新一轮的0号帧还是旧一轮重传的呢?接收方并不知道,很有可能接收方就把该0号帧当作新一轮的帧接收了,但实际上这个0号帧是重传的,所以出现了错误,即发送窗口的大小不可能为\textbf{2\textsuperscript{n}}。现在假设发送窗口的大小为\textbf{2\textsuperscript{n}}
-1,情况和上面一样,发送方发送了0~\textbf{2\textsuperscript{n}}
-2号帧,接收方发送的确认帧都丢失了,如果没有丢失就应该接着传\textbf{2\textsuperscript{n}}
-1号帧,但是丢失了,发送方应该发送0号帧。由于这种情况接收方可以判断出来(即下一帧只要不是第\textbf{2\textsuperscript{n}}
-1号帧就是重传),因此不会发生错误。如果发送窗口尺寸小于\textbf{2\textsuperscript{n}}
-1,那就更不会发生错误了。

\textbf{综上所述,后退N帧协议的最大发送窗口是\textbf{2\textsuperscript{n}}
-1。}
