\textbf{{1. 数据链路层的概念}}

数据链路层在物理层所提供服务的基础上向网络层提供服务,即\textbf{将原始的、有差错的物理线路改进成逻辑上无差错的数据链路},从而向网络层提供高质量的服务。它一般包括3种基本服务:\textbf{无确认的无连接服务、有确认的无连接服务和有确认的有连接服务。}

{\textbf{{记忆方式:}}有连接就一定要有确认,因为对方主机必须确认才可建立连接,即不存在无确认有连接服务)。}

\textbf{{2. 数据链路层的主要功能}}

\textbf{1)链路管理:}负责数据链路的建立、维持和释放,主要用于面向连接的服务。\\

\textbf{2)帧同步:}接收方应当能从接收到的二进制比特流中区分出帧的起始与终止。

\textbf{3)差错控制:}用于使接收方确定接收到的数据就是由发送方发送的数据。

\textbf{4)透明传输:}不管数据是什么样的比特组合,都应当能在链路上传送。
