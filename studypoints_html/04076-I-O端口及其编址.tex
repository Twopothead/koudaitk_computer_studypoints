{\textbf{1.I/O端口}}

\textbf{I/O端口}和\textbf{I/O接口}是两个不同的概念。{端口}指可以由CPU进行读或写的寄存器。这些寄存器分别用来存放数据信息、控制信息和状态信息,分别被称为数据端口、控制端口和状态端口。若干个端口加上相应的控制逻辑电路就组成了{接口}。

{\textbf{2.I/O端口的编址}}

I/O设备与主机交换信息和CPU与主存交换信息有很多的不同点,例如,CPU如何对I/O编址就是其中待解决的问题。

一般将I/O设备码看作\textbf{地址码}。

I/O地址码的编址一般采用两种方式:\textbf{统一编址和不统一编址}。

前面讲过了存储器地址,如果能将I/O的地址码和主存的地址码统一起来就方便多了,即{统一编址},例如,在64KB地址的存储空间中,划出8KB地址作为I/O设备的地址,凡是在这8KB地址范围内的访问,就是对I/O设备的访问,所用的指令和访存指令相似。显然,统一编址占用了存储空间,减少了主存容量,但无须专用的I/O指令。

{不统一编址}指I/O地址和存储器地址是分开的,所有对I/O设备的访问必须有专用的I/O指令。不统一编址由于不占用主存空间,故不影响主存容量,但需要设置I/O专用指令。因此,设计机器时,需根据实际情况衡量考虑选取何种编址方式。
