{为了避免操作系统及其关键数据(如PCB等)受到用户程序有意或无意的破坏,通常将处理器的执行状态分为两种:核心态与用户态。}

\textbf{{核心态:}又称管态、系统态,是操作系统管理程序执行时机器所处的状态。}它具有较高的特权,能执行包括特权指令的一切指令,能访问所有寄存器和存储区。

\textbf{{用户态:}又称目态,是用户程序执行时机器所处的状态,}是具有较低特权的执行状态,它只能执行规定的指令,访问指定的寄存器和存储区。

划分核心态与用户态之后,用户态程序不能直接调用核心态程序,而是通过执行访问核心态的命令,引起中断,由中断系统转入操作系统内的相应程序,例如,在系统调用时,将由用户态转换到核心态。

\textbf{特权指令:}只能由操作系统内核部分使用,不允许用户直接使用的指令,如I/O指令、设置中断屏蔽指令、清内存指令、存储保护指令、设置时钟指令。
