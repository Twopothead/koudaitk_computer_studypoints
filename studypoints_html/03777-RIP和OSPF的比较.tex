\textbf{{1. RIP和OSPF的比较}}

\textbf{{a. 协议参数}}

RIP中用于表示目的网络远近的参数为\textbf{跳数},且参数被限制为最大15。而OSPF协议,路由表中表示目的网络的参数为费用(如时延),该参数为一虚拟值,即OSPF路由信息不受物理跳数的限制。因此,\textbf{OSPF协议适合应用于大型网络}。

\textbf{{b. 收敛速度}}

RIP周期性地将整个路由表作为路由信息广播至网络中,该广播周期为30s。在一个较大型的网络中,RIP会产生很大的广播信息,占用较多的网络带宽资源,并且由于RIP
30s的广播周期,影响了RIP的收敛,甚至出现不收敛的现象。而OSPF是一种链路状态的路由协议,当网络比较稳定时,网络中的路由信息比较少,并且其广播也不是周期性的,因此\textbf{OSPF路由协议在大型网络中也能够较快地收敛}。

\textbf{{\includegraphics{file://C:/Users/ADMINI~1/AppData/Local/Temp/SGTpbq/4656/029033D9.png}c.~分层}}

在RIP中,网络是一个平面的概念,并无区域及边界等的定义。在OSPF路由协议中,一个网络或者一个自治系统可以\textbf{划分为很多个区域},每一个区域通过OSPF边界路由器相连。

\textbf{{d. 负载平衡}}

在OSPF路由选择协议中,如果到同一个目的网络有多条相同代价的路径,那么可以将通信量分配给这几条路径。这称为多路径间的\textbf{负载平衡}。而RIP不会,它只能按照一条路径传送数据。

\textbf{{e. 灵活性}}

\textbf{OSPF协议对于不同类型的业务可以计算出不同的路由,十分灵活}。而这种灵活性是RIP所没有的。

\textbf{{f. 以组播地址发送报文}}

RIP使用{\textbf{广播}}报文来发送给网络上的所有设备,而OSPF协议采用{\textbf{组播}}地址来发送,只有运行OSPF协议的设备才会接收发送来的报文,其他设备不参与接收。
