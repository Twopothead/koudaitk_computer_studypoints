\textbf{{1. TCP连接的建立}}

\textbf{{TCP的连接建立采用``三次握手''的方法},}{}目的是为了\textbf{防止报文段在传输连接建立过程中出现差错}。通过3次报文段的交互后,通信双方的进程之间就建立了一条传输连接,然后就可以用{\textbf{全双工}}的方式在该传输连接上正常地传输数据报文段了。

1)SYN=1,seq=x。\\
2)SYN=1,ACK=1,seq=y,ack=x+1。\\
3)ACK=1,seq=x+1,ack=y+1。

\textbf{\textbf{{2. TCP连接的释放}}\\
}

一旦数据传输结束,参与传输的任何一方都可以请求释放传输连接。在释放连接过程中,\textbf{发送端进程与接收端进程要通过4次TCP报文段来释放整个传输连接}。

1)FIN=1,seq=u。\\
2)ACK=1,seq=v,ack=u+1。\\
3)FIN=1,ACK=1,seq=w,ack=u+1。\\
4)ACK=1,seq=u+1,ack=w+1。

{\textbf{提醒:}}{\textbf{不管是连接还是释放,SYN、ACK、FIN的值一定是1。}}
