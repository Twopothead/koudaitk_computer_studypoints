{\textbf{1.I/O中断的定义}}

CPU启动I/O后,不必停止现行程序的运行。而I/O接到启动命令后,进入自身的准备阶段。当准备就绪时,向CPU提出请求,此时CPU立即中断现行程序(不看书了),并保存断点(用书签做记号),转至执行中断服务程序(打球去了),为I/O服务。中断服务程序结束后,CPU又返回到程序的断点处,继续执行原程序。\\
\textbf{一次中断处理的过程可简单归纳为5个阶段:}

\textbf{1)中断请求。}设置中断请求触发器。

\textbf{2)中断判优。}设置中断判优,用硬件或者软件都可以实现。

\textbf{3)中断响应。}

\textbf{4)中断服务。}

\textbf{5)中断返回。}

{\textbf{2.}}{\textbf{为了处理I/O中断,在I/O接口电路中必须配置如下相关的硬件线路:}}

\textbf{1)}中断请求触发器和中断屏蔽触发器。

\textbf{2)}排队器。

\textbf{3)}中断向量地址形成部件。

{\textbf{3.}}{\textbf{中断服务程序的完整流程。}}

\textbf{(1)保护现场}

保护现场有两个含义:其一是{保存程序的断点};其二是{保存通用寄存器和状态寄存器的内容}。前者由中断隐指令完成,后者可由中断服务程序完成。

\textbf{(2)中断服务 (设备服务)}

这是中断服务程序的主体部分,不同的中断请求源其中断服务操作内容是不同的。

\textbf{(3)恢复现场}

这是中断服务程序的结尾部分,要求在退出服务程序前,将原程序中断时的``现场''恢复到原来的寄存器中。

\textbf{(4)中断返回}

中断服务程序的最后一条指令通常是一条\textbf{中断返回指令},使其返回到原程序的断点处,以便继续执行原程序。\\
