\textbf{{1. 单一连续分配}}\\

a.
基本概念:单一连续分配是一种最简单的存储管理方式,通常只能用于单用户、单任务的操作系统中。\textbf{这种存储管理方式将内存分为两个连续存储区域,其中的一个存储区域固定地分配给操作系统使用,通常放在内存低地址部分,另一个存储区域给用户作业使用。}

b.
技术采用:\textbf{单一连续分配方式采用静态分配,适合单道程序,可采用覆盖技术。}{作业一旦进入内存,就要等到其结束后才能释放内存。因此,这种分配方式不支持虚拟存储器的实现,无法实现多道程序共享主存。}

{c.
优点:}{管理简单,只需要很少的软件和硬件支持,且便于用户了解和使用,不存在其他用户干扰的问题。}

d.
缺点:是只能用于单用户、单任务的操作系统,内存中只装入一道作业运行,从而导致各类资源的利用率都很低。{\textbf{单一连续分配会产生内部碎片。}}

\textbf{{2. 固定分区分配}}

a.
基本概念:固定分区分配(也称为固定分区存储管理)方法是最早使用的一种可运行多道程序的存储管理方法,它\textbf{将内存空间划分为若干个固定大小的分区,每个分区中可以装入一道程序};

b.
技术采用:固定分区分配中,程序通常采用\textbf{静态重定位方式装入内存};

c.~分区大小可以相等也可以不相等:

\textbf{分区大小相等:}缺乏灵活性,造成内存空间的浪费,当程序太大时,一个分区又不足以装入该程序,导致程序无法运行;

\textbf{分区大小不相等:}可把内存区划分成含有多个较小的分区、适量的中等分区及少量的大分区。可根据程序的大小为之分配适合的分区;

d. 优点:可用于多道程序系统最简单的存储分配;

e:
缺点:不能实现多进程共享一个主存区,利用率较低,{\textbf{会产生内部碎片}}。

\textbf{{3.动态分区分配}}

a.
基本概念:动态分区分配又称为可变式分区分配,是一种\textbf{动态划分存储器的分区}方法。这种分配方法并不事先将主存划分成一块块的分区,而是在作业进入主存时,\textbf{根据作业的大小动态地建立分区,并使分区的大小正好满足作业的需要。}因此,系统中分区的大小是可变的,分区的数目也是可变的。

b.
分区分配算法:主要有以下四种,分别是首次适应算法、下次适应算法、最佳适应算法、最差适应算法;

c.
优点:{实现了多道程序共用主存;管理方案相对简单;实现存储保护的手段比较简单。}

d.
缺点:主存利用不够充分,存在外部碎片;无法实现多进程共享存储器信息;无法实现主存的扩充,进程地址空间受实际存储空间的限制。

\textbf{{4.内部碎片与外部碎片}}

a. 内存碎片:{分配给进程的分区中未被利用的碎片称为内部碎片;}

b. 外部碎片:而系统中剩余的无法利用的小块存储空间称为外部碎片。
