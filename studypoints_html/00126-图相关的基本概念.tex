{\textbf{(1)图:}图是有结点的有穷集合V和边的集合E组成。}

{\textbf{(2)有向图和无向图:}有向图每条边都有方向,无向图每条边都没有方向。}

{\textbf{(3)弧:}有向图中,边称为弧,记为\textless{}v\textsubscript{i},v\textsubscript{j}\textgreater{},表示一条从顶点{v}\textsubscript{i}到顶点{v}\textsubscript{j}的边。}

{\textbf{(4)顶点的度、入度和出度:}与顶点v相关的边数称为顶点v的度。入度和出度都是有向图的概念。在有向图中指向顶点v的边数称为顶点v的入度,由顶点v发出的边数称为顶点v的出度。}

{\textbf{(5)有向完全图和无向完全图:}有向图最多有n(n-1)条边,称这样具有n(n-1)条边的有向图为有向完全图。无向图最多有n(n-1)/2条边,称这样具有n(n-1)/2条边的无向图为无向完全图。}

{\textbf{(6)路径和路径长度:}路径即为相邻顶点序偶构成的序列,如\textless{}a,b\textgreater{},\textless{}b,c\textgreater{}构成一条路径。路径长度指路径上边的数目。}

{\textbf{(7)简单路径:}路径序列中顶点和边都不重复出现的路径为简单路径。}

{\textbf{(8)回路:}一条路径第一个顶点和最后一个顶点相同,称为回路。}

{\textbf{(9)无向图的连通、连通图和连通分量:}这3个概念都是针对无向图的。从顶点{v}\textsubscript{i}到顶点{v}\textsubscript{j}有路径,称{v}\textsubscript{i}和{v}\textsubscript{j}连通。任意两个顶点之间都连通,成为连通图。否则,将其中最大的极大连通子图称为连通分量。}

{\textbf{(10)有向图的连通、强连通图和强连通分量:}这3个概念都是针对有向图的。从顶点{v}\textsubscript{i}到vj有路径,称{v}\textsubscript{i}和{v}\textsubscript{j}连通。对于每一对顶点{v}\textsubscript{i}和{v}\textsubscript{j},从{v}\textsubscript{i}到{v}\textsubscript{j}和{v}\textsubscript{j}到{v}\textsubscript{i}都有路径,称该图为强连通图。否则,将其中极大强连通子图称为强连通分量。}

{\textbf{(11)权和网:}图中每条边都可以有值,用于表示从一个顶点到另一个顶点的距离(代价),该数值称为权。{边上带权的图称为带权图,也称为网。}}
