\textbf{{1.共享存储器系统}}

为了传输大量数据,\textbf{{在存储器中划出一块共享存储区,多个进程可以通过对共享存储区进行读写来实现通信}}。进程在通信前,向系统申请建立一个共享存储区,并指定该共享存储区的关键字。若该共享存储区已经建立,则将该共享存储区的描述符返回给申请者。然后,申请者把获得的共享存储区附接到进程上。这样,进程便可以像读写普通存储器一样读写共享存储区了。

\textbf{{2.消息传递系统}}\\

在消息传递系统中,进程间以消息为单位交换数据,\textbf{{用户直接利用系统提供的一组通信命令(原语)来实现通信}}。操作系统隐藏了通信的实现细节,简化了通信程序,得到了广泛应用。根据实现方式不同,消息传递系统可以分为以下两类:

\textbf{a.
直接通信方式:}发送进程直接把消息发送给接收进程,并将它挂在接收进程的消息缓冲队列上,接收进程从消息缓冲队列中取得消息。

\textbf{b.
间接通信方式:}发送进程把消息发送到某个中间实体(通常称为信箱)中,接收进程从中取得消息。这种通信方式又称为信箱通信方式。该通信方式广泛应用于计算机网络中,与之相应的通信系统称为电子邮件系统。

\textbf{{3.管道通信系统}}\\

管道{\textbf{是用于连接读进程和写进程以实现它们之间通信的}}{\textbf{共享文件}},向管道提供输入的发送进程(即写进程)以字符流形式将大量的数据送入管道,而接收管道输出的进程(即读进程)可以从管道中接收数据。

\textbf{注意:管道是一个共享文件,不能单纯地从字面上仅将管道理解为一个传输通道。}
