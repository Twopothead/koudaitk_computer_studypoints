{\textbf{1.I/O系统的历史演变过程}}

\textbf{(1)早期阶段(程序查询方式)}

在早期阶段,主存、CPU、I/O系统都买不起房子,三个兄弟挤在一套房子里。CPU住在主存和I/O系统的中间。每次I/O系统和主存交换信息的时候都需要经过CPU才能交换完成。可想而知,这样极其浪费时间。

\textbf{(2)接口模块和DMA阶段}(中断方式和DMA方式)

在该阶段I/O设备通过接口模块与主机相连,且采用了\textbf{总线连接}的方式。虽然这仅仅是简单的一句话,却基本解决了以上的两个问题。

\textbf{(3)具有通道结构的阶段}

通道是用来负责管理I/O设备以及实现主存与I/O设备之间交换信息的部件,可以视为一种具有特殊功能的处理器。通道具有专用的通道指令,能独立地执行用通道指令所编写的输入/输出程序,但不是一个完全独立的处理器。它依据CPU的I/O指令进行启动、停止或改变工作状态,是从属于CPU的一个专用处理器。

\textbf{(4)具有I/O处理器的阶段}

该部分不属于计算机组成原理中的讲解内容,考研基本不会涉及。

\textbf{{2.I/O系统的基本概念}}

I/O系统主要由两部分组成:\textbf{I/O软件和I/O硬件。}

\textbf{(1)I/O软件}

对于采用接口模块方式,要使得I/O设备与主机协调工作,必须靠I/O指令来完成。对于采用通道管理方式,不仅需要I/O指令,还需要\textbf{通道指令}。

\textbf{(2)I/O硬件}

输入/输出系统的硬件组成是多种多样的,在带接口的I/O系统中,I/O硬件包括接口模块和I/O设备两大部分;在具有通道或I/O处理器的I/O系统中,I/O硬件包括通道(或称为处理器)、设备控制器和I/O设备等。\\
