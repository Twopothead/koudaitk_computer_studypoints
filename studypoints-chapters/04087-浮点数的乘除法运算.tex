\textbf{{1. 浮点数乘除法的运算规则}}

\textbf{运算规则:}两个浮点数相乘,乘积的阶码应为相乘两数的阶码之和,乘积的尾数应为相乘两数的尾数之积。两个浮点数相除,商的阶码为被除数的阶码减去除数的阶码,尾数为被除数的尾数除以除数的尾数所得的商,下面可以用数学公式来描述。

假设有两个浮点数x和y:

\includegraphics[width=0.92708in,height=0.15625in]{http://texmath.koudaitiku.com/cgi-bin/mathtex.cgi?sign=19580c\&x=Sx*r\%5E\%7Bjx\%7D}

\includegraphics[width=0.91667in,height=0.18750in]{http://texmath.koudaitiku.com/cgi-bin/mathtex.cgi?sign=85f579\&y=Sy*r\%5E\%7Bjy\%7D}

那么有,\\

\includegraphics[width=1.93750in,height=0.19792in]{http://texmath.koudaitiku.com/cgi-bin/mathtex.cgi?sign=38bd8a\&x*y=(Sx*Sy)*r\%5E\%7Bjx+jy\%7D}(1)

\includegraphics[width=1.23958in,height=0.38542in]{http://texmath.koudaitiku.com/cgi-bin/mathtex.cgi?sign=5f267f\&/frac\{x\}\{y\}=/frac\{Sx\}\{Sy\}*r\^{}\{jx-jy\}}(2)

从式1和式2可以看出,\textbf{浮点数乘除运算不存在两个数的对阶问题,故比浮点数的加减法还要简单。}

\textbf{{2. 浮点数乘除法的运算步骤}}

\textbf{第一步:0操作数检查。}

对于乘法:检测两个尾数中是否一个为0,若有一个为0,则乘积必为0,不再作其他操作;若两尾数均不为0,则可进行乘法运算。\\

对于除法:若被除数x为0,则商为0;若除数y为0,则商为
,另作处理。若两尾数均不为0,则可进行除法运算。

\textbf{第二步:阶码加减操作。}\\

在浮点乘除法中,对阶码的运算只有4种,即+1、-1、两阶码求和以及两阶码求差。当然,在运算的过程中,还要检查是否有溢出,因为两个同号的阶码相加或异号的阶码相减可能产生溢出。

\textbf{第三步:尾数乘/除操作。}\\

对于乘法:第2章讲解了非常多的定点小数乘法算法,两个浮点数的尾数相乘可以随意选取一种定点小数乘法运算来完成。

\textbf{第四步:结果规格化及舍入处理。}\\
可以直接采用浮点数加减法的规格化和舍入处理方式。主要有以下两种:\\

\textbf{1)第一种:无条件地丢掉正常尾数最低位之后的全部数值。}这种办法被称为截断处理,其好处是处理简单,缺点是影响结果的精度。
~

\textbf{2)第二种:运算过程中保留右移中移出的若干高位的值,最后再按某种规则用这些位上的值进行修正尾数。}这种处理方法被称为舍入处理。

{\textbf{当尾数用补码表示时}}{\textbf{具体规则是:}}

1)当丢失的各位均为0时,不必舍入。\\
2)当丢失的各位数中的最高位为0,且以下各位不全为0时,或者丢失的最高位为1,以下各位均为0时,舍去丢失位上的值。\\
3)当丢失的最高位为1,以下各位不全为0时,执行在尾数最低位加1的修正操作。\\
4)舍入处理。\\
