{\textbf{{一、显示器}}}

{\textbf{1.显示器的分类}}

按显示设备{所用的显示器件}分类:\textbf{阴极射线管(CRT)显示器}{(重点)、液晶显示器(LCD)、等离子显示器。}

按所{显示的信息内容}分类:{字符显示器、图形显示器、图像显示器。}

\textbf{2. 阴极射线管(CRT)显示器}

\textbf{(1)CRT显示器的分类}

CRT显示器按扫描方式的不同,可分为光栅扫描和随机扫描;按分辨率的不同,又可分为高分辨率和低分辨率。

\textbf{(2)分辨率和灰度级}

\textbf{分辨率}指显示器所能表示的像素个数,像素越密,分辨率越高,图像越清晰。

\textbf{灰度级}指黑白显示器中所显示的像素点的亮暗差别,在彩色显示器中则表现为颜色的不同,灰度级越高,图像层次越清楚逼真。

\textbf{(3)刷新和刷新存储器}

CRT发光是由电子束打在荧光粉上引起的,电子束扫过之后,其发光亮度只能维持几十毫秒便会消失。为了使人眼能看到稳定的图像显示,必须使电子束不断地重复扫描整个屏幕,这个过程称为刷新。按人的视觉生理,\textbf{刷新频率大于30次/s时才不会感到闪烁}。

为了不断提高刷新图像的信号,必须把图像信息存储在刷新存储器(也称为视频存储器)。其存储容量由图像分辨率和灰度级决定,分辨率越高,灰度级越高,刷新储存器容量越大。假如分辨率为\includegraphics[width=0.87500in,height=0.13542in]{texmath/0423981024times1024}像素,256级灰度的图像(需要8位二进制数来表示)\textbf{储存容量}为\includegraphics[width=2.30208in,height=0.16667in]{texmath/523f1a1024times1024timeslog22561MB}。

{\textbf{{二、打印机}}}

打印输出是计算机最基本的输出形式,与显示器输出相比,打印输出可产生永久性记录,因此打印设备又被称为硬拷贝设备。
