\textbf{{1.TCP的数据编号与确认}}

\textbf{a.
TCP协议是面向字节的。}TCP将所要传送的报文看成是\textbf{字节组成的数据流},并使每一个字节对应于一个序号。在连接建立时,双方要商定初始序号。

\textbf{b.
TCP的确认是对接收到的数据的最高序号表示确认。}因此,确认号表示接收端期望下次收到的数据中的第一个数据字节的序号。

% file://C:/Users/ADMINI~1/AppData/Local/Temp/SGTpbq/4656/02FB0730.png}

\textbf{c.~在后退N帧协议中,}
如果帧不按序到达,直接丢弃后面的,没有使用选择确认。

\textbf{d.
在选择重传协议中,}先把后面有序的帧存在接收缓冲区中,等到前面失序的帧到达后,一起按序交付,这里就用到选择确认SACK。

\textbf{{2.TCP的重传机制}}

TCP每发送一个报文段,就对这个报文段设置一次计时器。\textbf{{只要计时器设置的重传时间到了规定时间还没有收到确认,那么就要重传该报文段}}。TCP采用了一种自适应的算法,如下:

{\textbf{第一步:}记录每个报文段发出的时间以及收到相应的确认报文段的时间。这两个时间之差就是报文段的往返时延。}

{\textbf{第二步:}将各个报文段的往返时延样本加权平均,就得出报文段的平均往返时延(}{RTT}{)。}

{\textbf{第三步:}}{每测量到一个新的往返时延样本,就按下式重新计算一次平均往返时延。}

{}

{RTT}={(1}-a{)\includegraphics[width=0.09375in,height=0.08333in]{texmath/ff0c81times}}{(}旧的{RTT)}+a\includegraphics[width=0.09375in,height=0.08333in]{texmath/ff0c81times}{(}新的往返时延样本{)}

{ }

{在上式中,0≤a\textless{}1。若a很接近于1,表示新算出的平均往返时延RTT和原来的值相比变化较大,即RTT的值
更新较快。若选择a接近于0,则表示加权计算的RTT受新的往返时延样本的影响不大,即RTT的值更新较慢,一般推荐a取0.125。}

{计时器的超时重传时间(RTO)应略大于上面得出的RTT,即}{RTO=\includegraphics[width=0.10417in,height=0.14583in]{texmath/338478beta}\includegraphics[width=0.09375in,height=0.08333in]{texmath/ff0c81times}RTT(其中\includegraphics[width=0.10417in,height=0.14583in]{texmath/338478beta}\textgreater{}1)}
