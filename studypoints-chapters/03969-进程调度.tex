\textbf{{1. 进程调度的功能}}

{a. 记录系统中所有进程的有关情况以及状态特征;}

{b. 选择获得处理器的进程;}

{c. 处理器分配。}

\textbf{{2. 引起进程调度的原因({重点之重点})}}

\textbf{a.~}当前运行进程运行结束;

\textbf{b.~}当前运行进程因某种原因从运行状态进入阻塞状态;

\textbf{c.~}执行完系统调用等系统程序后返回用户进程;

\textbf{d.~}在采用抢占调度方式的系统中,一个具有更高优先级的进程要求使用处理器;\\

\textbf{e.~}在分时系统中,分配给该进程的时间片已用完。

\textbf{{3.进程调度的方式}}\\

进程调度方式是指当某一个进程正在处理器上执行时,若有某个更为重要或紧迫的进程需要进行处理(即有优先级更高的进程进入就绪队列),此时应如何分配处理器。通常有两种进程调度方式。

\textbf{抢占方式:}又称为可剥夺方式。这种调度方式是指{当一个进程正在处理器上执行时,若有某个优先级更高的进程进入就绪队列,则立即暂停正在执行的进程},将处理器分配给新进程。

\textbf{非抢占方式:}又称为不可剥夺方式。这种方式是指{当某一个进程正在处理器上执行时,即使有某个优先级更高的进程进入就绪队列,仍然让正在执行的进程继续执行,直到该进程完成或因发生某种事件而进入完成或阻塞状态时},才把处理器分配给新进程。
