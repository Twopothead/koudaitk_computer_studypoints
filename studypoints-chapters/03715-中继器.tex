物理层设备主要包含\textbf{中继器}和\textbf{集线器},当然还有其他设备,但考研只需掌握此两种即可。

在计算机网络中,最简单的就是两台计算机通过两块网卡构成双机互连,这两台计算机的网卡之间一般是由非屏蔽双绞线来充当信号线的。由于双绞线在传输信号时信号功率会逐渐衰减,当信号衰减到一定程度时会造成信号失真,因此在保证信号质量的前提下,双绞线的最大传输距离为100m。当两台计算机之间的距离超过100m时,为了实现双机互连,人们便在这两台计算机之间安装一个中继器,它的作用就是将已经衰减得不完整的信号经过整理,重新产生出完整的信号再继续传送。

{\textbf{注意:放大器和中继器都是起放大信号的作用,只不过放大器放大的是模拟信号,中继器放大的是数字信号。}}
