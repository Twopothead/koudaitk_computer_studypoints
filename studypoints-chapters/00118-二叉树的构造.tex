{同一棵二叉树(结点值均不相同)具有唯一的先序、中序、后序和层次序列,但不同的二叉树可能具有相同的先序、中序、后序和层次序列。{二叉树的构造就是根据提供的某些遍历序列构造二叉树的结构。}}

{关于二叉树的构造有以下几种方式,构造方法都是用递归的思想:}

{\textbf{1. 由先序序列和中序序列构造二叉树}}

{{先序:根结点(左子树的先序)(右子树的先序);中序:(左子树的中序)根结点(右子树的中序)。这样就能确定出根结点,同时又得到了左子树的先序和中序,右子树的先序和中序。继续找左右子树的根结点,直到子树的结点个数为1。构造完成。}\\
}

{\textbf{2. 由后序序列和中序序列构造二叉树}}

{{后序:(左子树的后序)(右子树的后序)根结点;中序:(左子树的中序)根结点(右子树的中序)。这样就能确定出根结点,同时又得到了左子树的后序和中序,右子树的后序和中序。继续找左右子树的根结点,直到子树的结点个数为1。构造完成。}\\
}

{\textbf{3. 由层次序列和中序序列构造二叉树}}

{这个举例会更清晰些,例如中序为DGBAECF,层次遍历为ABCDEFG。由层次遍历得知,根结点为A。再根据中序就等到DGB为左子树的中序,ECF为右子树的中序。又子树的层次遍历的结点顺序跟整棵树的层次遍历的结点顺序一致。得到左子树的层次遍历为BDG,右子树的层次遍历为CEF。这样就知道了左子树的层次和中序,右子树的层次和中序,采用同样方法即可,构造出整个二叉树。}
