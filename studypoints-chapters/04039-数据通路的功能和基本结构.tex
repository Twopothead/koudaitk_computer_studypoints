{\textbf{1.数据通路的功能}}

建立数据通路的功能就是实现CPU内部的运算器和寄存器,以及寄存器之间的数据交换。

{\textbf{2.数据通路的基本结构}}

\textbf{数据通路的基本结构主要有以下两种方式:}

\textbf{1)CPU内部总线方式。}将所有寄存器的输入端和输出端都连接到一条或多条公共的通路上,这种结构比较简单,但是数据传输存在较多的冲突现象,性能较低。如果连接各部件的总线只有一条,则称为单总线结构。如果CPU中有两条或多条总线,则构成双总线结构和多总线结构。在双总线或多总线结构中,数据的传递可以同时进行。

\textbf{2)专用数据通路方式。}根据指令执行过程中的数据和地址的流动安排连接线路,避免使用共享的总线,性能比较高,但硬件量较大。
