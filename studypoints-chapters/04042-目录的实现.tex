\textbf{{1.线性表}}

\textbf{最为简单的目录实现方法是使用存储文件名和数据块指针的线性表(数组、链表等)}{。创建新文件时,必须首先搜索目录表以确定没有同名的文件存在,接着在目录表后增加一个目录项。若要删除文件,根据给定的文件名搜索目录表,接着释放分配给它的空间。采用链表结构可以减少删除文件的时间,其优点在于实现简单,不过由于线性表需要采用顺序方法查找特定的项,故运行比较费时。}

\textbf{{2.散列表}}

\textbf{散列表根据文件名得到一个值,并返回一个指向线性表中元素的指针。}这种方法大大缩短了查找目录的时间,插入和删除也比较简单,不过需要一些措施来避免冲突(两个不同名文件的散列函数值相同)。这种方法的特点是散列表长度固定以及散列函数对表长的依赖性。
