\textbf{{1.磁盘格式化}}

{一个新的磁盘只是一个含有磁性记录材料的空白盘。在}\textbf{磁盘能存储数据前,它必须分成扇区以便磁盘控制器能进行读和写操作,这个过程称为低级格式化。}{低级格式化为磁盘的每个扇区采用独特的数据结构。每个扇区的数据结构通常由头、数据区域(通常为512B)和尾部组成。头部和尾部包含了一些磁盘控制器所使用的信息。}{为了使用磁盘存储文件,操作系统还需要将自己的数据结构记录在磁盘上:}

\textbf{a.
将磁盘分为由一个或多个柱面组成的分区}(就是常见的C盘、D盘等分区)。\\

\textbf{b.
对物理分区进行逻辑格式化}(创建文件系统),操作系统将初始的文件系统数据结构存储到磁盘上,这些数据结构包括空闲和已经分配的空间以及一个初始为空的目录。

\textbf{{2.引导块}}

计算机启动时需要运行一个初始化程序(自举程序),它初始化CPU、寄存器、设备控制器和内存等,接着启动操作系统。为此,该\textbf{自举程序应找到磁盘上的操作系统内核,装入内存,并转到初始地址,从而开始操作系统的运行。}

自举程序通常保存在ROM中,为了避免改变自举代码需要改变ROM硬件的问题,只在ROM中保留很小的自举装入程序,将完整功能的自举程序保存在磁盘的启动块上,启动块位于磁盘的固定位。\textbf{拥有启动分区的磁盘称为启动磁盘或系统磁盘。}

\textbf{{3.坏块}}

{由于硬件有移动部件且容错能力差,因此容易导致一个或多个扇区损坏。根据所使用的磁盘和控制器,对这些块有多种处理方式。}

a. 简单磁盘:坏扇区可手工处理。\\
b.
复杂的磁盘:其控制器维护一个磁盘坏块链表。该链表在出厂前进行低级格式化时就初始化了,并在磁盘的整个使用过程中不断更新。低级格式化将一些块保留作为备用,对操作系统透明。控制器可以用备用块来逻辑地替代坏块,这种方案称为\textbf{{扇区备用}。}
