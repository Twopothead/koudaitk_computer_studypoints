{\textbf{1.DMA方式的特点}}

DMA方式是一种完全由硬件进行成组信息传送的控制方式,具有程序中断方式的优点,即在数据准备阶段,CPU与外设并行。它还降低了CPU在传送数据时的开销,这是因为\textbf{信息传送不再经过CPU,而在外设与内存之间直接进行,}因此称为直接存储器存取方式。由于数据传送不经过CPU,也就不需要保护、恢复CPU现场等烦琐操作。这种方式适用于磁盘、磁带等高速设备大批量数据的传送。

{\textbf{2.DMA的传送方法}}

如果出现高速I/O(通过DMA接口)和CPU同时访问主存怎么办?这个时候CPU就得将总线(如地址线、数据线等)的占有权让给DMA接口使用,即DMA采用\textbf{周期窃取}的方式占用一个存储周期。通常,DMA与主存交换数据时采用以下3种方法。

\textbf{(1)停止CPU访问主存}

当外设需要传送一片数据时,由DMA接口向CPU发一个信号,要求CPU放弃地址线、数据线和有关控制线的使用权,DMA接口获得总线控制权后,开始进行数据传送。在数据传送结束后,DMA接口通知CPU可以使用主存,并把总线控制权交还给CPU。在这种传送过程中,\textbf{CPU基本处于不工作状态或保持原始状态}。

\textbf{(2)周期挪用}

当I/O设备没有DMA请求时,CPU按程序的要求访问主存,一旦I/O设备有DMA请求,就会遇到3种情况:第一种情况是{CPU不在访存}(CPU正在执行加法指令),故I/O的访存请求与CPU未发生冲突;第二种情况是{CPU正在访存,}必须等待存储周期结束后,CPU再将总线占有权让出;第三种情况是{I/O和CPU同时请求访存},出现了访存冲突,此刻CPU要暂时放弃总线占有权,由I/O设备挪用一个或几个存储周期。

\textbf{(3)DMA与CPU交替访问}

这种方式适用于CPU的工作周期比主存存取周期长的情况,例如,CPU的工作周期是1.2s,主存的存取周期小于0.6s,那么可将一个CPU周期分为C1和C2两个周期,其中C1专供DMA访存,C2专供CPU访存。

{\textbf{3.DMA接口的功能}}

利用DMA方式传送数据时,数据的传输过程完全由DMA接口电路控制,故DMA接口又称为DMA控制器。{DMA接口应具有以下几个功能:}

\textbf{1)}向CPU申请DMA数据传送。

\textbf{2)}在CPU允许DMA工作时,处理总线控制权的转交,避免因进入DMA工作而影响CPU正常活动或引起总线竞争。

\textbf{3)}在DMA期间管理系统总线,控制数据传送。

\textbf{4)}确定数据传送的起始地址和数据长度,并修正数据传送过程中的数据地址和数据长度。

\textbf{5)}在数据块传送结束时,给出DMA操作完成的信号。

DMA的传送过程分为\textbf{预处理、数据传送和后处理}3个阶段。
