预防死锁的发生,只需破坏死锁产生的4个必要条件之一即可。下面具体分析与这4个条件相关的技术。

\textbf{{1.互斥条件}}\\

{{为了破坏互斥条件,}}{就要}\textbf{{允许多个进程同时访问资源}}。但是这会受到资源本身固有特性的限制,有些资源根本不能同时访问,只能互斥访问,如打印机就不允许多个进程在其运行期间交替打印数据,只能互斥使用。由此看来,通过破坏互斥条件来防止死锁的发生是不大可能的。

\textbf{{2.不剥夺条件}}\\

为了破坏不剥夺条件,可以制定这样的策略:对于一个已经获得了某些资源的进程,若\textbf{{新的资源请求不能立即得到满足,则它必须释放所有已经获得的资源,以后需要资源时再重新申请}}。

\textbf{{3.请求与保持条件}}\\

为了破坏请求与保持条件,可以采用预先静态分配方法。预先静态分配法{\textbf{要求进程在其运行之前一次性申请它所需要的全部资源,在它的资源未满足前,不把它投入运行}}。一旦投入运行后,这些资源就一直归它所有,也不再提出其他资源请求,这样就可以保证系统不会发生死锁。

\textbf{{4.环路等待条件}}\\

为了破坏环路等待条件,可以采用有序资源分配法\textbf{。}有序资源分配法是{\textbf{将系统中的所有资源都按类型赋予一个编号(例如打印机为1,磁带机为2),要求每一个进程均严格按照编号递增的次序请求资源,同类资源一次申请完}}。
