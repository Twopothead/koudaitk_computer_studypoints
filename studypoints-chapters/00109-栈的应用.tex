以下部分只是让考生明白,什么问题可以用栈来解决。

\textbf{{1. 数制转换}}

如十进制转换为八进制的程序,计算过程是从低位到高位顺序产生八进制的各个位数。\textbf{{而打印过程是从高位到低位进行,恰好和计算过程相反}}。因此,若将计算过程中得到的八进制数的各位依次入栈,最后按出栈序列打印输出即为与输入对应的八进制数。

\textbf{{2. 括号匹配的检验}}

考虑下列括号序列:\\
\hspace*{0.333em} ~ ~~{[} ( {[} {]} {[} {]} ) {]}

计算机接收了第一个左中括号后,期待右中括号的出现。等来的确是左小括号,此时第一个中括号必须等待,左小括号被消掉之后,才有可能被消掉,即这里\textbf{{满足了栈的``先进后出''的特点}}。用栈的处理括号的匹配时,遇左括号就进栈,遇右括号就出栈。

\textbf{{3. 行编辑程序}}

在编辑程序中,设立了一个缓冲区用于接收用户输入的一行字符,然后逐行存入用户数据区。允许用户输入出差错,并在发现有误时可以及时更正。例如用``\#''表示退格符,用``@''表示退行符。其中退格符,删除的是最后输入的字符,\textbf{{符合栈后入先出的特点}}。退行符,删除的是缓冲区全部元素。这里用栈实现很方便。{遇到普通字符,入栈。遇到退格符,栈顶元素出栈。遇到退行符,清空栈。}

\textbf{{4. 迷宫求解}}

在``穷举求解''迷宫时,为了保证在任何位置上都能沿着原路退回,显然需要用一个后进先出的结构来保存从入口到当前位置的路径。因此,在求解迷宫通路的算法中应用``栈''也就是很自然的事情。

\textbf{{5. 表达式求值}}

算法基本思想如下:

(1)首先置操作数栈为空栈,表达式起始符``\#''为运算符栈的栈底元素(为了和表达式的结束符,构成整个表达式的一对括号)。

(2)依次读入表达式每个字符,若是操作数则进OPND栈,若是运算符则和OPTR栈栈顶运算符比较优先权后作相应操作,直至整个表达式求解完毕(即OPTR栈的栈顶元素和当前读入的字符均为``\#'')。

\textbf{{6. 递归的实现(知道就行)}}
