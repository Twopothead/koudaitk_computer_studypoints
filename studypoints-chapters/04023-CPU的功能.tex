{程序一旦被送出存储器,控制器就要开始工作了,负责协调并控制计算机各部件执行程序的指令序列,}\textbf{其基本功能是取指令、分析指令和执行指令。}{除了以上三大基本功能外,控制器还必须能控制程序的输入和运算结果的输出(即控制主机与I/O设备交换信息)以及对总线的管理,甚至能处理机器运行过程中出现的异常情况(如掉电)和特殊请求(如打印机请求打印一行字符),即处理中断的能力。}{\textbf{控制器的功能总结如下:}}

\textbf{1)}控制器能自动地形成指令的地址,并能发出取指令的命令,将对应此地址的指令取到控制器中,称为\textbf{指令控制}。

\textbf{2)}取到指令之后,应该产生完成每条指令所需要的控制命令,称为\textbf{操作控制}。

\textbf{3)}控制命令产生后,需要对各种控制命令加以时间上的控制,称为\textbf{时间控制}。

\textbf{4)}在执行的过程中,可能需要进行算术运算和逻辑运算,称为\textbf{数据加工}。

\textbf{5)}最后当然还有处理中断的能力,称为\textbf{中断处理}。
