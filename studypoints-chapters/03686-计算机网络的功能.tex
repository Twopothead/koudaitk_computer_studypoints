\textbf{1.
数据通信:}是计算机网络最基本和最重要的功能,包括连接控制、传输控制、差错控制、流量控制、路由选择、多路复用等子功能。

\textbf{2. 资源共享:}包括数据资源、软件资源以及硬件资源。

\textbf{3.
分布式处理:}当计算机网络中的某个计算机系统负荷过重时,可以将其处理的任务传送给网络中的其他计算机系统进行处理,利用空闲计算机资源提高整个系统的利用率。

\textbf{4.
信息综合处理:}将分散在各地计算机中的数据资料进行集中处理或分级处理,如自动订票系统、银行金融系统、数据采集与处理系统等。

\textbf{5. 负载均衡:}将工作任务均衡地分配给计算机网络中的各台计算机。

\textbf{6. 提高可靠性:}计算机网络中的各台计算机可以通过网络互为替代机。

当然,为了满足人们的学习、工作和生活需要,计算机网络还有其他一些功能,如远程教育、电子化办公与服务、娱乐等。
