{\textbf{1. 设备控制器的组成}}

{设备控制器由}\textbf{设备控制器与处理器的接口}{、}\textbf{设备控制器与设备的接口}{及}\textbf{I/O逻辑}{三部分组成。}

{\textbf{2. 设备控制器的功能}}

a. 接收和识别来自CPU的各种指令;\\
b. 实现CPU与设备控制器、设备控制器与设备之间的数据交换;\\
c. 记录设备的状态供CPU查询;\\
d. 识别所控制的每个设备的地址。

{\textbf{3. 程序直接控制方式}}

a.
基本概念:早期的计算机系统中,没有中断系统,所以CPU和I/O设备进行通信、传输数据时,由于CPU的速度远远快于I/O设备,因此\textbf{{CPU需要不断地测试I/O设备}},这种控制方式又称为轮询或忙等。\\
b. 优点:程序直接控制方式的工作过程非常简单。\\
c. 缺点:CPU的利用率相当低。

\textbf{{4.中断控制方式}}

a.
基本概念:{\textbf{CPU与外设并行工作,CPU不用等待外设}},当外设准备好数据后通过中断通知CPU,CPU停下当前的工作来处理外设的数据。\\
b.
优点:与程序直接控制方式相比,有了中断的硬件支持后,CPU和I/O设备间可以并行工作了,CPU只需收到中断后处理即可,大大提高了CPU利用率。\\
c.
缺点:但这种控制方式仍然存在一些问题,例如每台设备每输入输出一个数据,都要求中断CPU,这样在一次数据传送过程中的中断次数过多,从而消耗了大量CPU时间。\\
d. 中断处理程序的处理过程如下:\\
1)唤醒被阻塞的驱动(程序)进程\\
2)保护被中断进程的CPU环境\\
3)转入想要的设备处理程序\\
4)中断处理\\
5)恢复被中断进程的现场

\textbf{{5.DMA控制方式}}

{a.
基本概念:}{在外设和内存之间开辟直接的数据交换通路}{。在DMA控制方式中,设备控制器具有更强的功能,在其控制下,设备和内存之间可以成批地进行数据交换,而不用CPU干预。}

b.
特点:数据传输的基本单位是数据块,而且数据是单向传输,从设备到内存或者相反。

% \includegraphics{file://C:/Users/ADMINI~1/AppData/Local/Temp/SGTpbq/4140/1024A7E6.png}
c.~优点:{DMA控制方式下,设备和CPU可以并行工作,同时设备与内存的数据交换速度加快,并且不需要CPU干预。}

d.
缺点:DMA控制方式仍然存在一定局限性,如数据传送的方向、存放输入数据的内存起始地址及传送数据的长度等都由CPU控制,并且每台设备都需要一个DMA控制器,当设备增加时,多个DMA控制器的使用也不经济。

\textbf{{6.通道控制方式}}

a.
基本概念:是一种以内存为中心,实现设备与内存直接交换数据的控制方式。与DMA控制方式相比,通道所需要的CPU干预更少,而且可以做到一个通道控制多台设备,从而更进一步减轻了CPU负担。\textbf{通道本质上是一个简单的处理器},通道的指令系统比较简单,\textbf{一般只有数据传送指令、设备控制指令}等。\\
b.~优点:{解决了I/O操作的独立性和各部件工作的并行性。通道把中央处理器从繁琐的输入输出操作中解放出来。采用通道技术后,不仅能实现CPU和通道的并行操作,而且通道与通道之间也能实现并行操作,各通道上的外设也能实现并行操作,从而可达到提高整个系统效率的根本目的。\\
c.
缺点:由于需要更多硬件(通道处理器),因此成本较高。通常应用于大型数据交互的场合。\\
d.
I/O通道与一般处理器的区别:I/O通道的指令类型单一,其所能执行的命令主要局限于与I/O操作有关的指令;通道没有自己的内存,通道所执行的通道程序放在主机的内存中,也就是说通道是与CPU共享内存的。}
