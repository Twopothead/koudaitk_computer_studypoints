{\textbf{1. FDDI环}}

a.
FDDI环称为\textbf{光纤分布式数据接口}。只需要把FDDI环看成是令牌环形网络的一种就行了,只不过其传输介质是多模光纤以及其他一些功能的改进。

b. FDDI环仍然基于IEEE
802.5令牌环标准的MAC协议,上层仍采用与其他局域网相同的逻辑链路控制LLC协议,分组最大长度为4500B。FDDI环主要是用做\textbf{校园环境的主干网。}

c.
FDDI使用了比令牌环更复杂的方法访问网络。和令牌环一样,也需在环内传递一个令牌,而且允许令牌的持有者发送FDDI帧。和令牌环不同,FDDI网络可在环内传送多个帧。因为令牌接受了传送数据帧的任务以后,FDDI令牌持有者可以立即释放令牌,把它传给环内的下一个站点,无需等待数据帧完成在环内的全部循环。这意味着,第一个站点发出的数据帧仍在环内循环的时候,下一个站点可以立即开始发送自己的数据,故FDDI网络可在环内传送多个帧。

\textbf{{注意:}由光纤构成的FDDI环,其基本结构为逆向双环。一个环为主环,另一个环为备用环。一个顺时针传送信息,另一个逆时针传送信息。当主环上的设备失效或光缆发生故障时,通过从主环向备用环的切换可继续维持FDDI的正常工作。这种故障容错能力是其他网络所没有的。}
