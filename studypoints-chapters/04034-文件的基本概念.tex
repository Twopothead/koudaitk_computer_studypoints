\textbf{{1.文件的概念}}

{在计算机中,}\textbf{大量的数据和信息是通过文件存储和管理的}{。文件系统负责管理文件,并为用户提供对文件进行}\textbf{存取、共享及保护}{的方法。}

文件:\textbf{有文件名的一组相关元素的集合},在文件系统中是一个最大的数据单位,它描述了一个对象集,每个文件都有一个文件名,用户通过文件名来访问文件。

\textbf{{2.文件的属性}}

{{a. 名称:文件名唯一,以容易读取的形式保存;}\\
{b. 标识符:系统内文件的唯一标签,通常为数字,对用户来说是透明的;}\\
{c. 文件类型:被支持不同类型的文件系统所使用;}\\
{d. 文件位置:指向文件的指针;}\\
{e. 文件的大小、建立时间、用户标识等。}}

\textbf{{3.文件的分类}}

a. 按照用途分类:系统文件、库文件、用户文件;\\
b. 按保护级别分类:只读文件、读写文件、执行文件、不保护文件;\\
c. 按信息流向分类:输入文件、输出文件、输入输出文件。\\
d. 按数据形式分类:源文件、目标文件、可执行文件。

\textbf{{4.文件的操作}}

a.
基本的文件操作:创建文件、删除文件、读文件、写文件、截断文件、设置文件的读/写位置等。

b.
文件的打开操作:{\textbf{是指系统将文件的属性从外存复制到内存,并设定一个编号(或索引)返回给用户}。以后当用户要对该文件进行操作时,只需利用编号(或索引号)向系统提出请求即可。这样避免了系统对文件的再次检索,节约了检索开销,也提高了对文件的操作速度。}\\
c.
文件的关闭操作:\textbf{是指系统将打开的文件的编号(或索引号)删除,{并销毁其文件控制块}}。如果文件被修改,则需要将修改保存到外存。
