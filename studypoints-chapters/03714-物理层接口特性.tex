{\textbf{1. 物理层接口特性概念}}

{物理层应\textbf{{尽可能地屏蔽各种物理设备的差异}},使得数据链路层只需考虑本层的协议和服务。换句话说,物理层主要的功能其实就是\textbf{{确定与传输介质的接口有关的一些特性}},即物理层接口的特性。}

{\textbf{2. 物理层的四个特性}}{\textbf{}}

\textbf{机械特性:}指明接口的形状、尺寸、引线数目和排列等;

\textbf{电气特性:}电压的范围,即何种信号表示电压0和1;

\textbf{功能特性:}接口部件的信号线(数据线、控制线、定时线等)的用途。

\textbf{规程特性(2012年真题已考):}或称为{\textbf{过程特性}}{,物理线路上对不同功能的各种可能事件的出现顺序,即时序关系。}
