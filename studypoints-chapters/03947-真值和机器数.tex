日常生活中我们经常看到+5、-8、-0.1、+3.6等这些带有``+''或者``-''符号的数,称为{\textbf{真值}}。那么如果要用计算机处理这些数,计算机不认识``+''或者``-''符号怎么办?此时需要具备的思维能力:\textbf{一想到具有两种状态的事物,都应该联想到二进制的0和1。}

恰好``+''``-''是两种状态,于是就可以使用二进制的0和1来表示。那就再做一个规定:\textbf{0表示正号,1表示负号}。这样就可以将一个真值完全数字化了,而被数字化的数就称为机器数(机器数分为原码、补码、反码和移码)。
