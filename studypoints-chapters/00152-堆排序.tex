{堆是一种数据结构,{可以把堆看成是一个完全二叉树},这个完全二叉树满足:{任何一个非叶子结点的值,都不大于(或小于)其左右孩子结点的值,称之为小顶堆(或大顶堆)。}}

{将无序序列调整成一个堆后,就可以找出这个序列的最小(或最大)值,然后将其交换到序列的最前(或最后),这样这个被交换的最值就排好序了。对新的无序序列重复这样的操作,就实现了排序。因此堆排序中最关键的操作是建堆和调整堆。之前认真看二叉树存储的考生,应该还记得完全二叉树适合用一个数组来存储,而且左右孩子的下标都是可以直接根据父结点下标算出来的。}

{\textbf{}}{\textbf{算法分析:}时间复杂度为}{O(nlog\textsubscript{2}n)}{,空间复杂度为}{O(1)}{。}{\\
}
