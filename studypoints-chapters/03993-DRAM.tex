\textbf{{1.DRAM的工作原理}}

DRAM分为三部分即\textbf{保持存储信息、读数据和写数据。}

\textbf{{2.DRAM存储器的刷新}}

\textbf{通常有3种刷新方式:集中刷新、分散刷新和异步刷新。}

\textbf{{集中刷新}}

把刷新操作集中到一段时间内集中进行(集中``歼灭'')。一般来说,电容上的电荷基本只能维持2ms,在2ms内必须要刷新一次。将2ms(2000us)看成一个刷新周期。假设存取周期为0.5us,那么在一个刷新周期里有4000个存取周期。假设该存储矩阵有32行,则对32行集中刷新需要16us。刷新的时候是不能进行读/写操作的,故称刷新这段时间为``死时间'',又称为访存``死区''。可以计算出死区的占用比例为32/4000=0.8\%(称为``死时间''率)。

\textbf{{分散刷新}}

将刷新操作分散进行,周期性地进行(分散``歼灭'')。分散刷新需要重点讲解,因为里面有一个不太容易理解的知识点。在分散刷新中,存储周期已经不再是传统的存储周期了。也就是说,此时存储周期不再等于读(写)周期,而这里扩展了操作的定义,即:

\textbf{{存储周期=读或写周期+刷新一行的时间}}

从这个公式可以看出,此时存储周期为读或写周期的两倍。此处刷新一行的时间又看成是等于存取周期的。

\textbf{{异步刷新}}\\

是一个折中方案,既不会像集中刷新那么大费周章,产生集中的固定时间,也不会像分散刷新那么频繁地刷新,而是有计划地刷新,时间分配十分合理。异步刷新是把存储矩阵的每行分散到2ms时间内刷新,但不是集中刷新,而是\textbf{{平均地分配}}{。}这就能保证当刷新完第一行后,再过2ms又可完成对第一行的下一次刷新。这样就不会像分散刷新那样每个存取周期都刷新某一行。
