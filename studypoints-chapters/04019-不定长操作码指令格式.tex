{\textbf{不定长操作码指令格式就是操作码的长度不固定,}}操作码的长度随地址码个数的减少而增加,不同的地址数的指令可以具有不同长度的操作码。这样,在满足需要的前提下,有效地缩短了指令字长。在设计操作码指令格式时,必须注意以下两点:

\textbf{1)不允许较短的操作码是较长操作码的前缀,}例如,某条指令的操作码是11,而另外一条指令的操作码是111,11是111的前缀,这样译码就会出现歧义,故不采纳(可以联系数据结构中的赫夫曼树编码来理解)。

\textbf{2)各条指令的操作码一定不可以重复。}

通常情况下,对使用频率较高的指令,分配较短的操作码;对使用频率较低的指令,分配较长的操作码,从而尽可能减少指令译码和分析的时间。
