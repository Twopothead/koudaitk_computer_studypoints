\textbf{{1. 以太网}}

a. 知识背景:IEEE
802.3标准{\textbf{是一种基带总线型的局域网标准}}。在不太严格区分的时候,IEEE
802.3可以等同于以太网标准,因为它是基于原来的以太网标准诞生的一个总线型局域网标准\textbf{。}{随着快速以太网、千兆以太网和万兆以太网相继进入市场,{\textbf{以太网现在几乎成了局域网的同义词}}。}

b.
工作原理:{\textbf{以太网采用总线拓扑结构}},所有计算机都共享一条总线,{\textbf{信息以广播方式发送}}。为了保证数据通信的方便性和可靠性,以太网使用了CSMA/CD技术对总线进行访问控制。考虑到局域网信道质量好,以太网采取了以下{两项重要的措施}以使通信更加简便。

1)采用无连接的工作方式。\\

2)不对发送的数据帧进行编号,也不要求对发送方发送确认。

因此以太网提供的服务是{\textbf{不可靠的服务}},即尽最大努力交付,差错的纠正由传输层的TCP完成。
