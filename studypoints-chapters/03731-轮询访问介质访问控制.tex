轮询访问介质访问控制主要用在令牌环局域网中,目前使用得很少。在轮询访问介质访问控制中,用户不能随机地发送信息,而是通过一个集中控制的监控站经过轮询过程后再决定信道的分配。典型的轮询访问介质访问控制协议就是\textbf{{令牌传递协议}}{。}

\textbf{令牌环局域网}把多个设备安排成一个物理或逻辑连接环。为了确定哪个设备可以发送数据,让一个令牌(特殊格式的帧)沿着环形总线在计算机之间依次传递。当计算机都不需要发送数据时,令牌就在环形网上``游荡'',而需要发送数据的计算机只有拿到该令牌才能发送数据帧,所以不会发生冲突(因为令牌只有一个),这就是所谓的受控接入。
