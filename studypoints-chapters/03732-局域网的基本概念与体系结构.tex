局域网(Local Area
Network,LAN)\textbf{{是指一个较小范围(如一个公司)内的多台计算机或者其他通信设备,通过双绞线、同轴电缆等连接介质互连起来}},以达到资源和信息共享目的的互联网络。

\textbf{{1.局域网最主要的特点}}

\textbf{a.~}局域网为一个单位所拥有(如学校的一个系使用一个局域网)。

\textbf{b.~}地理范围和站点数目有限({\textbf{双绞线的最大传输距离为100m,如果要加大传输距离,则在两段双绞线之间安装中继器}},最多可安装4个中继器,例如,安装4个中继器连接5个网段,则最大传输距离可达500m,所以地理范围有限。局域网一般可以容纳几台至几千台计算机,所以站点数目有限)。

\textbf{c.~}与以前非光纤的广域网相比,{\textbf{局域网具有较高的数据率、较低的时延和较小的误码率}}(现在局域网的数据率可以达到万兆了;传输距离较短所以时延小;距离短了失真就小,误码率自然就低)。

\textbf{{2.局域网的主要优点}}

\textbf{a.~}{具有广播功能},从一个站点可很方便地访问全网。局域网上的主机可共享连接在局域网上的各种硬件和软件资源。

\textbf{b.~}便于系统的扩展和演变,各设备的位置可灵活地调整和改变。

\textbf{c.~}提高了系统的可靠性、可用性。

\textbf{{3.局域网的主要技术要素}}

局域网的主要技术要素包括{\textbf{网络拓扑结构、传输介质与介质访问控制方法}}。其中,介质访问控制方法是最为重要的技术特性,决定着局域网的技术特性。

\textbf{{4.局域网的主要拓扑结构}}

局域网的主要拓扑结构包括{\textbf{星形网、环形网、总线型网和树形网}}。

\textbf{{5.局域网的主要传输介质}}

局域网的主要传输介质{\textbf{包括双绞线、铜缆和光纤等,其中双绞线为主流传输介质}}。

\textbf{{6.局域网的主要介质访问控制方法}}

局域网的主要介质访问控制方法{\textbf{包括CSMA/CD、令牌总线和令牌环}}。
