\textbf{{1. 透明网桥(选择的不是最佳路由)}}

a.
基本概念:透明网桥是目前使用最多的网桥。``透明''是指局域网上的站点并不知道所发送的帧将经过哪几个网桥,因为网桥对各站来说是看不见的。\textbf{{透明网桥是一种即插即用设备}},意思是只要把网桥接入局域网,不用人工配置转发表,网桥就可以开始工作。{既然上面提到了网桥可以不用人工配置转发表,网桥如何自学习?}

b. 自学习步骤 :

\textbf{1)网桥收到一帧后先进行自学习。}查找转发表中与收到帧的源地址有无相匹配的项目。若没有,就在转发表中增加一个项目(源地址、进入的接口和时间)。若有,则把原有的项目进行更新。

\textbf{2)转发帧。}查找转发表中与收到帧的目的地址有无相匹配的项目。若没有,则通过所有其他接口(但进入网桥的接口除外)进行转发。若有,则按转发表中给出的接口进行转发。{若转发表中给出的接口是该帧进入网桥的接口,则应丢弃这个帧(因为这时不需要经过网桥进行转发)。}

\textbf{{2. 源选径网桥(选择最佳路由)}}

在源选径网桥中,路由选择由发送数据帧的源站负责,网桥只根据数据帧中的路由信息对帧进行接收和转发。

为了发现合适的路由,{\textbf{源站}}{\textbf{先以}}{\textbf{广播方式向欲通信目的站发送一个发送帧}}。发送帧将在整个局域网中沿着所有可能的路由传送,并记录所经过的路由。当发送帧到达目的站时,就沿原来的路径返回源站。源站在得知这些路由后,再从所有可能的路由中选择一个最佳路由。
