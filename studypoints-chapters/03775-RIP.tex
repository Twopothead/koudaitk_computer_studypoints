\textbf{{1.距离-向量算法}}

RIP认为一个好的路由就是它通过的路由器的数目少,即``距离短''。RIP允许{\textbf{一条路径最多只能包含15个路由器。``距离''的最大值为16时即相当于不可达}}。可见,RIP只适用于小型互联网({\textbf{因为比较大的自治系统里面的路由器的数量肯定会大大超过15个}})。RIP不能在两个网络之间同时使用多条路由。RIP选择一个具有最少路由器的路由(即最短路由),哪怕还存在另一条高速(低时延)但路由器较多的路由。

{\textbf{2. RIP的三要点}}

1)仅和相邻路由器交换信息。

2)交换的信息是当前本路由器所知道的全部信息,即自己的路由表。\\

3)按固定的时间间隔(如每隔30s)交换路由信息。

{\textbf{注意:}{RIP选择的路径不一定是时间最短的,但一定是具有最少路由器的路径。因为它是根据最少的跳数进行路径选择的。}}

\textbf{{3.RIP的优缺点}}

\textbf{a. 优点:}实现简单、开销小,收敛过程较快。

\textbf{b. 缺点:}

1)RIP限制了网络的规模,它能使用的最大距离为15(16表示不可达)。

2)路由器之间交换的路由信息是路由器中的完整路由表,因而随着网络规模的扩大,开销也就增加了。

3)当网络出现故障时,RIP要经过比较长的时间才能将此信息传送到所有的路由器,即``坏消息传播得慢'',使更新过程的收敛时间长。
