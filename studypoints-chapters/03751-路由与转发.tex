{\textbf{1. 路由器的两大核心功能}}

{a.
两大核心功能:}\textbf{路由选择(确定哪一条路径)}{和}\textbf{分组转发(当一个分组到达时所采用的动作)}{。}

% \includegraphics{file://C:/Users/ADMINI~1/AppData/Local/Temp/SGTpbq/8544/14DE1503.gif}
\textbf{b.路由}{选择:根据路由算法}\textbf{{确定一个进来的分组应该被传送到哪一条输出路线上}}{。如果子网内部使用数据报,那么对每一个进来的分组都要重新选择路径。如果子网内部使用虚电路,那么只有当创建一个新的虚电路时,才需要确定路由路径。}

c.
分组转发:就是\textbf{{路由器根据转发表将用户的IP数据报从合适的端口转发出去}}。

\textbf{{\textbf{{提醒1:}}{路由表是根据路由选择算法得出的,而转发表是从路由表得出的。}{}}}

\textbf{{\textbf{{\textbf{提醒2:}}}}}{路由器可在网络之间转发分组(即IP数据报)。特别是,这些互联的网络可以是异构的。}因此{,如果是许多相同类型的网络互连在一起,那么用一个很大的交换机(如果能够找得到)代替原来的一些路由器是可以的。\textbf{如果这些互联的网络是异构的网络,那么就必须使用路由器来进行互联}。}
