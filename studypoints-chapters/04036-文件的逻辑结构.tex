\textbf{{1. 无结构的流式文件}}

a.
基本概念:由若干个字符组成,\textbf{可以看做是一个字符流,称为流式文件}。可以将流式文件看成记录式文件的特例。在UNIX系统中,所有文件都被视为流式文件,系统不对文件进行格式处理。

{\textbf{\textbf{{2. 有结构的记录式文件(顺序文件)}}}}

{a.
基本概念:顺序文件}{又称为连续结构,是一种最简单的文件结构,其}\textbf{将一个逻辑文件的信息连续存放}{。以顺序结构存放的文件称为顺序文件或连续文件。}

{b.
分类:}{按照记录}\textbf{是否定长}{,顺序文件分为}\textbf{定长记录顺序文件}{和}\textbf{变长记录顺序文件。}{按照文件中记录}\textbf{是否按照关键字排序}{,顺序文件又分}\textbf{为串结构和顺序结构}{:串结构中各记录之间的顺序与关键字无关,而顺序结构中所有记录按照关键字顺序排序。}

c.
优点:顺序文件\textbf{存取时速度较快};当文件为定长记录文件时,还可以根据文件起始地址及记录长度进行\textbf{随机访问}。

d.
缺点:因为文件存储要求连续的存储空间,所以会产生碎片,同动{时也不利于文件的}{态扩充。}

\textbf{\textbf{{3. 有结构的记录式文件(索引文件)}}}

{a.
基本概念:}\textbf{索引结构为一个逻辑文件的信息建立一个索引表。}{索引表中的表目存放文件记录的长度和所在逻辑文件的起始位置,因此逻辑文件中不再保存记录的长度信息。索引表本身是一个定长文件,每个逻辑块可以是变长的,索引表和逻辑文件两者构成了索引文件。}

b. 优点:可以进行随机访问,也易于进行文件的增删。

c.
缺点:索引表的使用增加了存储空间的开销,另外,索引表的查找策略对文件系统的效率影响很大。

\textbf{{\textbf{\textbf{{4. 有结构的记录式文件(索引顺序文件)}}}}}

{a.
基本概念:}\textbf{索引顺序文件是顺序和索引两种形式的结合。}{索引顺序文件将顺序文件中的所有记录分为若干个组,为顺序文件建立一张索引表,并为每组中的第一个记录在索引表建立一个索引项,其中含有该记录的关键字和指向该记录的指针。}

索引表中包含\textbf{关键字和指针}两个数据项,索引表中索引项按照关键字顺序排列。索引顺序文件的逻辑文件(主文件)是一个顺序文件,每个分组内部的关键字不必有序排列,但是组与组之间的关键字是有序排列的。\\

b. 优点:{大大提高了顺序存取的速度。}

c. 缺点:仍然需要配置一个索引表,增加了存储开销。

\textbf{{5.直接文件和散列(Hash)文件}}

{a.
基本概念:}\textbf{建立关键字和相应记录物理地址之间的对应关系,}{这样就可以}\textbf{直接通过关键字的值找到记录的物理地址}{,也就是说,关键字的值决定了记录的物理地址,这种结构的文件称为直接文件。这种映射结构不同于顺序文件或索引文件,没有顺序的特性。}

b.
优点:通过散列函数对关键字进行转换,转换结果直接决定记录的物理地址。{散列文件\textbf{有很高的存取速度}。}

c. 缺点:由于不同关键字的散列函数值相同而引起冲突。
