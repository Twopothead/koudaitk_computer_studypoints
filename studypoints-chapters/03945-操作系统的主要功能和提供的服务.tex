\textbf{{一、操作系统的主要功能}}

\textbf{{1.处理器管理}}

{进程控制:}负责进程的创建、撤销及状态转换。

{{进程同步}{:}}{对并发执行的进程进行协调。}

{进程通信:}负责完成进程间的信息交换。\\

{进程调度:}按一定算法进行处理器分配。

{\textbf{2.存储器管理}}

{内存分配:}按一定的策略为每道程序分配内存。

{内存保护:}保证各程序在自己的内存区域内运行而不相互干扰。

{内存扩充:}为允许大型作业或多作业的运行,{}必须借助虚拟存储技术去获得增加内存的效果。

\textbf{{3.设备管理}}

{设备分配:}根据一定的设备分配原则对设备进行分配。

{设备传输控制:}实现物理的输入输出操作,即启动设备、中断处理、结束处理等。

{设备独立性:}即用户程序中的设备与实际使用的物理设备无关。

\textbf{{4.文件管理}}\\

文件存储空间的管理:负责对文件存储空间进行管理,包括存储空间的分配与回收等功能。

{目录管理:}目录是为方便文件管理而设置的数据结构,它能提供按名存取的功能。

{文件操作管理:}实现文件的操作,负责完成数据的读写。

{文件保护:}提供文件保护功能,防止文件遭到破坏。

{\textbf{5.用户接口}}\\

{命令接口:}提供一组命令供用户直接或间接控制自己的作业。

{程序接口:}也称为系统调用,是程序级的接口,由系统提供一组系统调用命令供用户程序和其他系统程序调用。

{图形接口:}近年来出现的图形接口(也称图形界面)是命令接口的图形化。

\textbf{{二、操作系统提供的服务}}

由操作系统的功能可以知道操作系统提供哪些服务:操作系统提供了一个用以执行程序的环境,提供的服务有程序执行、I/O操作、文件操作、资源分配与保护、错误检测与排除等。
