\textbf{{可靠传输与无差错接收的区别总结。}}\\

解析:在数据链路层若仅仅使用循环冗余码检验差错检测技术,只能做到对帧的无差错接收,即``凡是接收端数据链路层接收的帧,都能以非常接近于1的概率认为这些帧在传输过程中没有产生差错''。接收端的帧虽然收到了,但最终还是因为有差错被丢弃,即没有被接收。

以上所述可以近似地表述为``凡是接收端数据链路层接收的帧均无差错''。

{注意:现在并没有要求数据链路层向网络层提供``可靠传输''的服务。}所谓``可靠传输'',就是数据链路层的发送端发送什么,接收端就接收什么。传输差错可分为两大类,一类就是比特差错(可以通过CRC来检测),而另一类传输差错更复杂,这就是收到的帧并没有出现比特差错,但却出现了\textbf{帧丢失}(如发送1、2、3,收到1、3)、\textbf{帧重复}(如发送1、2、3,收到1、2、2、3)、\textbf{帧失序}(如发送1、2、3,收到1、3、2)。这3种情况都属于``出现传输差错'',但都不是这些帧里有``比特差错''。

帧丢失很容易理解,但是帧重复、帧失序的情况较为复杂,在这里暂不讨论,学完可靠传输的工作原理后,就会知道在什么情况下接收端可能会出现帧重复或帧失序。

总之,\textbf{``无比特差错''和``无传输差错''并不是同样的概念},在数据链路层使用CRC检验只能实现无比特差错的传输,但这还不是可靠传输。
