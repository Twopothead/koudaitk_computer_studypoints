\textbf{{1. 拥塞控制}}

\textbf{a.
条件:}{出现资源拥塞的条件是}{对}\textbf{资源需求的总和\textgreater{}可用资源}{;}

\textbf{b.
后果:}若网络中产生拥塞,网络的性能就要明显变坏,整个网络的吞吐量将随输入负荷的增大而下降

\textbf{{2. 拥塞控制与流量控制的区别}}

\textbf{1)}拥塞控制所要做的只有一个前提,就是{\textbf{使得网络能够承受现有的网络负荷}}。

\textbf{2)}拥塞控制{\textbf{是一个全局性的过程,涉及所有的主机、所有的路由器}}以及与降低网络传输性能有关的所有因素。

\textbf{3)}流量控制往往指在\textbf{{给定的发送端和接收端之间的点对点}}通信量的控制。

\textbf{4)}流量控制所要做的就是抑制发送端发送数据的速率,以便使接收端来得及接收。

\textbf{5)}拥塞控制是很难设计的,因为它{\textbf{是一个动态的(而不是静态的)问题}}。

\textbf{6)}当前网络正朝着高速化的方向发展,这很容易出现缓存不够大而造成分组的丢失。但分组的丢失是网络发生拥塞的征兆而不是原因。

\textbf{7)}在许多情况下,甚至正是拥塞控制本身成为引起网络性能恶化甚至发生死锁的原因,这点应特别引起重视。

\textbf{{3. 拥塞控制的分类}}

\textbf{1)开环控制:}在设计网络时{\textbf{事先将有关发生拥塞的因素考虑周到}},力求网络在工作时不产生拥塞。

\textbf{2)闭环控制:}是{\textbf{基于反馈环路的概念}}。属于闭环控制的有以下几种措施:

① 监测网络系统以便检测到拥塞在何时、何处发生;

② 将拥塞发生的信息传送到可采取行动的地方;

{③ 调整网络系统的运行以解决出现的问题。}
