\textbf{{1.~}{访问的页在主存中,且访问页在快表中}}

{在快表中就表明在内存中,则}\textbf{EAT}{=查找快表时间+根据物理地址访存时间}\textbf{={a+t}。}

{\textbf{2. 访问的页在主存中,但不在快表中}}

{\textbf{EAT}=查找快表时间+查找页表时间+修改快表时间(题目未给出则忽略不计,如果给出,通常与访问快表时间相同)+根据物理地址访存时间\textbf{={a+t+a+t=2(a+t)}。}}

\textbf{{3. 访问的页不在主存中}}

既然不在主存中,就肯定不在快表中,即发生缺页,设处理缺页中断的时间为T(包括将该页调入主存,更新页表和快表的时间),则EAT=查找快表时间+查找页表时间+处理缺页时间(通常包括了更新页表和快表时间)+查找快表时间+根据物理地址访存时间\textbf{={a+t+T+a+t=T+2(a+t)}。}

接下来加入缺页率和命中快表几率,将上述3种情况组合起来,形成完整的有效访问时间计算公式。假设命中快表的几率为d,缺页率为f,则:\\

\textbf{EAT}=查找快表时间+d×根据物理地址访存时间+(1-d)×{[}查找页表时间+f×(处理缺页时间+查找快表时间+根据物理地址访存时间)+(1-f)×(修改快表时间+根据物理地址访存){]}\textbf{={a+d×t+
(1-d){[}t+f(T+a+t)+(1-f)(a+t){]}}。}

\textbf{{4. 快表访问和修改时间}}\\

有些题目会说明系统中有快表,如果没说明,则视为没有快表,将上述公式的命中率和访问时间变为0就可以了;有些题目会说明忽略访问和修改快表时间,则将访问时间a变为0就可以了。

\textbf{{5. 关于处理缺页中断时间T的计算}}

如果题目中没有说明被置换出的页面是否被修改,则缺页中断时间统一为一个值T。如果题目中说明了被置换的页面分为修改和未修改两种不同的情况,假设被修改的概率为n,处理被修改的页面的时间为T1,处理未被修改的页面的时间为T2,则\textbf{{处理缺页时间T=n×T1+
(1-n)×T2}。}
