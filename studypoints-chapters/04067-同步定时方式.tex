同步定时方式指系统采用一个{\textbf{统一的时钟信号}}来协调发送和接收双方的传送定时关系。\textbf{时钟信号通常由CPU的总线控制器部件发出},然后送到总线上的所有部件。

\textbf{同步通信的优点:}传送速度快,具有较高的传输速率。

\textbf{同步通信的缺点:}同步通信必须按最慢的模块来设计公共时钟,当总线上的模块存取速度差别很大时,便会大大损失总线效率,并且不知道被访问的外部设备是否真正响应,故可靠性较低。

\textbf{同步通信的使用范围:}总线长度较短、总线所接部件的存取时间应该都比较接近。
