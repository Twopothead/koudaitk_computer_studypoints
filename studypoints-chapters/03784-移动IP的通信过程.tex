\textbf{{1. 移动IP技术的通信过程}}

\textbf{1)}移动结点在本地网时,按传统的TCP/IP方式进行通信(在本地网有固定的地址)。

\textbf{2)}移动结点漫游到一个外地网络时,仍然使用固定的IP地址进行通信。为了能够收到通信对端发给它的IP分组,\textbf{移动结点需要向本地代理注册当前的位置地址},这个位置地址就是\textbf{转交地址}。移动IP的转交地址可以是外部代理的地址或动态配置的一个地址。

\textbf{3)}本地代理接收来自转交地址的注册后,会构建一条通向转交地址的隧道,将截获的发给移动结点的IP分组通过隧道送到转交地址处。

\textbf{4)}在转交地址处解除隧道封装,恢复出原始的IP分组,最后送到移动结点,这样移动结点在外网就能够收到这些送给它的IP分组了。

\textbf{5)}移动结点在外网通过外网的路由器或者外代理向通信对端发送IP数据报。

\textbf{6)}当移动结点来到另一个外网时,只需要向本地代理更新注册的转交地址,就可以继续通信了。

\textbf{7)}当移动结点回到本地网时,移动结点向本地代理注销转交地址,这时移动结点又将使用传统的TCP/IP方式进行通信。

{\textbf{总结:}}移动IP为移动主机设置了两个IP地址,即主地址和辅地址(转交地址)。移动主机在本地网时,使用的是主地址。当移动到另外一个网络时,需要获得一个辅助的临时地址,但是此时主地址仍然保持不变。当从外网移回本地网时,辅地址改变或撤销,而主地址仍然保持不变。\\
