\textbf{{1. CIDR}}

{a.
作用:无分类编址(CIDR)是{\textbf{{为解决IP地址耗尽而提出的一种措施}}}。CIDR消除了传统的A类、B类和C类地址以及划分子网的概念,因而可以更加有效地分配IPv4的地址空间。}

b.
地址格式:{CIDR使用各种长度的``网络前缀''来代替分类地址中的网络号和子网号。于是,IP地址又从三级编址回到了两级编址,其地址格式为}

\textbf{IP地址::={}

为了区分网络前缀,通常采用``斜线记法''(又称CIDR记法),即IP地址/网络前缀所占位数。

\textbf{{2. CIDR地址块}}

将网络前缀都相同的连续的IP地址组成``CIDR地址块''。{一个CIDR地址块可以表示很多地址,这种地址的聚合常称为\textbf{{{路由聚合(也称构成超网)}}},它使得路由表中的一个项目可以表示很多个原来传统分类地址的路由,因此可以缩短路由表,减小路由器之间选择信息的交换,从而提高网络性能。CIDR中同样使用了掩码来确定其网络前缀。对于``/20''的地址块,其掩码由连续的20个``1''和后续12个``0''组成。可以看出,``1''对应于网络前缀,``0''对应于主机号。}

在使用CIDR时,\textbf{{路由表中的每个项目由网络前缀和下一跳地址组成。}}这样就会导致查找路由表时可能会得到不止一个匹配结果。{\textbf{{应当从匹配结果中选择具有最长网络前缀的路由,因为网络前缀越长,其地址块就越小,路由就越具体}}}。最长前缀匹配原则又称为最长匹配或最佳匹配。
