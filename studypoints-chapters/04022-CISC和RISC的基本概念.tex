\textbf{1.RISC的主要特点总结}

{1)}{选取简单指令,让\textbf{复杂指令由简单指令的组合}来实现;}

{2)}{指令长度\textbf{固定},指令格式\textbf{种类少},寻址方式\textbf{种类少;}}

{3)}{\textbf{只有取数/存数指令访问存储器;}}

{4)}{CPU中\textbf{有多个通用寄存器;}}

{5)}{\textbf{采用流水线技术;}}

{6)}{控制器\textbf{采用组合逻辑控制;}}

{7)}{采用优化的编译程序。}

{\textbf{2.CISC的主要特点总结}}

1)\textbf{指令系统复杂庞大};

2)指令长度\textbf{不固定},指令格式\textbf{种类多},寻址方式\textbf{种类多;}

3)可以\textbf{访存的指令不受限制;}

4)由于80CISC\textbf{各指令的使用频率差距太大}。

5)各种\textbf{指令执行时间相差很大},大多数指令需多个时钟周期才能完成。

6)控制器大多数\textbf{采用微程序控制}。

7)难以用优化编译生成高效的目标代码程序。

{\textbf{3.RISC与CISC的比较(了解即可)}}

\textbf{1)}RISC比CISC更能提高计算机的运算速度,例如,由于RISC寄存器多,因此就可以减少访存次数,其次,由于指令数和寻址方式少,因此指令译码较快。

\textbf{2)}RISC比CISC更便于设计,可降低成本,提高可靠性。

\textbf{3)}RISC能有效支持高级语言程序。\\
