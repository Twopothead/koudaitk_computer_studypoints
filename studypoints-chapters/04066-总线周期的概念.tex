通常将完成一次总线操作的时间称为{\textbf{总线周期}},其可分为以下4个阶段:

\textbf{1)申请分配阶段。}由需要使用总线的主模块(或主设备)提出申请,经总线仲裁机器决定下一个传输周期的总线使用权授予某一申请者。

\textbf{2)寻址阶段。}取得了使用权的主模块通过总线发出本次要访问的从模块(或从设备)的地址及有关命令,启动参与本次传输的从模块。

\textbf{3)传送数据阶段。}主模块和从模块进行数据交换,数据由源模块发出,经数据总线流入目的模块。

\textbf{4)结束阶段。}主模块的有关信息均从系统总线上撤除,让出总线使用权。

如果该系统只有一个主模块,就无须申请、分配和撤除了,总线使用权始终归它所有。

总线通信控制主要解决通信双方如何获知传输开始和传输结束,以及通信双方如何协调和如何配合。通常,{\textbf{总线通信方式分为4类:}}同步通信(同步定时方式)、异步通信(异步定时方式)、半同步通信和分离式通信。\\
