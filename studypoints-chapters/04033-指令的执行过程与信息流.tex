{\textbf{(1)取指周期}}

\textbf{取指周期需要解决两个问题:}一个是\textbf{CPU到哪个存储单元取指令};另一个是\textbf{如何形成后继指令地址}。指令的地址由程序计数器(PC)给出。因此,取指周期的操作为:按PC内容取出指令,并将PC内容递增。当出现转移情况时,指令地址在执行周期被修改。

取指周期信息流如下:

1. \textbf{(PC)→MAR} ~ ~ ~ ~//将要执行指令的地址放到地址缓冲寄存器\\
2. \textbf{1→R} ~ ~ ~ ~ ~
~//发出读命令(固定写法),但是这个也可以不写,后面会详细讲解这种细节问题\\
3. \textbf{M(MAR)→MDR} ~ ~
//将要执行的指令从存储器中读到数据缓冲寄存器,其中(MAR)表示地址缓冲寄存器中的内容,所以M(MAR)就表示在主存中此地址的内容,即欲执行指令本身\\
4. \textbf{(MDR)→IR} ~ ~ ~ //将要执行的指令打入指令寄存器\\
5. \textbf{OP(IR)→CU} ~ ~
//(IR)表示指令本身,OP(IR)表示指令的操作码,AD(IR)表示指令的地址码\\
6. \textbf{(PC)+1→PC} ~ ~ ~//形成下一条指令的地址

{\textbf{(2)间址周期}}({并不是所有指令的执行过程中都会有间址周期})

\textbf{间址周期是为了}取出操作数的有效地址,操作数的地址存放在指令所对应的存储器(或者寄存器)中,然后到其所对应的存储器中去取操作数。

间址周期信息流如下:

1. \textbf{AD(IR)→MAR} ~ ~ ~
//将指令字中的地址码(形式地址)打入地址缓冲寄存器\\
2. \textbf{1→R} ~ ~ ~ ~ ~ ~ ~ ~ ~ ~//发出读命令\\
3. \textbf{M(MAR)→MDR ~} ~ //将有效地址从主存打入数据缓冲寄存器

{\textbf{(3)执行周期}}

不同指令的执行周期操作命令不一样,所以没有统一的格式。

{\textbf{(4)中断周期}}

执行周期结束后,{CPU}需要查询是否有请求中断的事件发生,如果有则进入中断周期。中断隐指令保存的断点存在哪里?怎么寻找中断服务程序入口地址?只有这两个问题确定了,才能写出微指令序列。

{前提:}现假设程序断点保存至主存的``0''号单元,且采用硬件向量法寻找入口地址。中断周期的微指令序列如下:

1. \textbf{0→MAR} ~//将主存``0''号单元的地址送入主存地址寄存器\\
2. \textbf{1→W} ~ ~ ~//启动存储器写\\
3.\textbf{(PC)→MDR}~ ~ ~ //将PC的内容(程序断点)送入主存数据寄存器\\
4.\textbf{(MDR)→M(MAR)} ~
~//将主存数据寄存器的内容写入MAR所指示的主存单元\\
5. \textbf{向量地址→PC} ~ ~ ~//将向量地址形成部件的输出送至PC\\
6. \textbf{0→EINT ~} ~ ~ ~//关中断,将允许中断触发器清零\\
以上就是中断周期的全部微指令操作。如果断点不是存入主存,而是存入堆栈,那么微程序指令又是什么?很简单,只需将上述步骤1改为:\\
1.\textbf{(SP)-1→SP,且(SP)→MAR ~
~~}//这里假设先修改指针,后存入数据
