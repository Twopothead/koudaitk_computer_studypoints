要掌握主存储器与CPU的连接之前,首先需要了解一个考研必考的知识点,即\textbf{{存储器容量的扩充。}}{一般存储器容量的扩充通常有3类:}\textbf{位扩充、字扩充和字位扩充}{。}

\textbf{1.
位扩充(增加{a}\includegraphics[width=0.09375in,height=0.08333in]{texmath/ff0c81times}{b}后面的b)}

位扩充指增加存储字长,例如,现要将1K\textbf{\includegraphics[width=0.09375in,height=0.08333in]{texmath/ff0c81times}}4位的芯片组成1K\textbf{\includegraphics[width=0.09375in,height=0.08333in]{texmath/ff0c81times}}8位的存储器。

\textbf{2.
字扩充(增加{a}\includegraphics[width=0.09375in,height=0.08333in]{texmath/ff0c81times}{b}前面的a)}

字扩充是增加存储单元的个数,例如,现要将1K\textbf{\includegraphics[width=0.09375in,height=0.08333in]{texmath/ff0c81times}}8位的芯片组成2K\textbf{\includegraphics[width=0.09375in,height=0.08333in]{texmath/ff0c81times}}8位的存储器。

\textbf{3.
字位扩充(增加{a}\includegraphics[width=0.09375in,height=0.08333in]{texmath/ff0c81times}{b}中的a和b)}

字位扩充既增加了存储单元的个数,又增加了存储字长,例如,要用1K\textbf{\includegraphics[width=0.09375in,height=0.08333in]{texmath/ff0c81times}}4位的芯片组成4K\textbf{\includegraphics[width=0.09375in,height=0.08333in]{texmath/ff0c81times}}8位的存储器。

{\textbf{注意:在进行字位扩充时,一定是先进行位扩充,再进行字扩充。}}

{ \textbf{} }

\textbf{{\textbf{另外,记住下面这个芯片数量计算公式:}}}

\textbf{}

{如果要求将容量为a}\includegraphics[width=0.09375in,height=0.08333in]{texmath/ff0c81times}{b的芯片组成容量为c}\includegraphics[width=0.09375in,height=0.08333in]{texmath/ff0c81times}{d的芯片,假设需要芯片的数量为n,则n=(c*d)/(a*b)(该公式就是整个存储器的容量除以单个芯片的容量)}。

以上掌握之后,只需要根据需要组成的芯片计算地址线和数据线的数量,再使用译码器(根据芯片的数量来选择,一般有2-4译码器和3-8译码器)连接起来即可。
