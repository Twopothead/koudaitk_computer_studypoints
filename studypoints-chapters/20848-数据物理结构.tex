{数据的物理结构(或存储结构)是数据的逻辑结构在计算机中的物理表示。它包括}{数据的表示和关系的表示。比如一个链表结点,结点包含值域和指针域。值域就是}{数据的表示,指针域就是关系的表示。}

在数据结构中有4种常用的存储方法:

1. {顺序存储方法:}把逻辑上相邻的结点存储在物理位置上相邻的存储单元中。

{2.
{链式存储方法:}不要求逻辑上相邻的结点在物理位置上相邻,结点间的逻辑关系是由附加的指针字段表示的。}

{3. {索引存储方法:}通过建立附加的索引表来标识结点的地址。}

{4.
{散列(或哈希)存储方法:}根据结点的关键字直接计算出该结点的存储地址。}
