在计算机中,最常用的字符编码是ASCII码。\textbf{{基本的ASCII码字符集共有128个字符,其中有96个可打印字符,}}包括常用的字母、数字、标点符号等,另外还有32个控制字符。考生有必要记住一些基本的编码,如常用数字的ASCII码,以及大小写字母的ASCII码。其实,字母和数字的ASCII码的记忆非常简单,只要记住一个字母或数字的ASCII码(如记住A为65,0的ASCII码为48),知道\textbf{相应的大小写字母之间差32,}就可以推算出其余字母或数字的ASCII码。

\textbf{虽然标准ASCII码是7位编码,}但由于计算机基本处理单位为字(1B=8bit),因此一般仍以一个字节来存放一个ASCII字符。每一个字节中多余出来的一位(最高位)在计算机内部通常保持为0(在数据传输时可用做奇偶校验位)。

\textbf{除ASCII码以外,}还有一些其他的常用编码:扩展ASCII码(对ASCII码的扩充,字节的最高位保持为1)、EBCDIC编码(一些大型主机系统如MVS、OS/390等使用的编码)、GB2312编码(GB代表``国标'',中国的标准化组织设计的简体汉字编码)、Unicode编码(``通用''编码,是统一了很多编码的一个大整合)等。\\

\textbf{在了解字符编码的基础上,}就不难理解字符串编码了。简单来说,字符串就是字符的``集合'',在计算机的存储中,通常在存储器中占用连续的多个字节空间,每个字节存储一个字符(若是汉字字符串,则是两个字节存储一个汉字)。有一个情况需要知道,当主存字由2个或4个字节组成时,在同一个主存字中,既可按从低位字节向高位字节的顺序存放字符串的内容,也可按从高位字节向低位字节的顺序存放字符串的内容,这个取决于使用的机器(\textbf{在第4章将会详细讲解高低字节的区别,即大小端})。

{注意:2012年的考研真题就针对小端方式进行了考查。}\\
