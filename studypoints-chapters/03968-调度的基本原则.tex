不同调度算法有不同的调度策略,这也决定了调度算法对不同类型的作业影响不同。在选择调度算法时,必须考虑不同算法的特性。为了衡量调度算法的性能,人们提出了一些评价标准。

\textbf{{1.CPU利用率}}\\

CPU是系统最重要、也是最昂贵的资源,其利用率是评价调度算法的重要指标。

\textbf{{2.系统吞吐量}}\\

系统吞吐量表示单位时间内CPU完成作业的数量。对长作业来说,由于它要占用较长的CPU处理时间,因此会导致系统吞吐量下降,而对短作业来说则相反。

\textbf{{3.响应时间}}\\

在交互系统中,尤其在多用户系统中,多个用户同时对系统进行操作,都要求在一定时间内得到响应,不能使某些用户的进程长期得不到调用。

\textbf{{4.周转时间}}\\

从每个作业的角度来看,完成该作业的时间是至关重要的,通常用周转时间或者带权周转时间来衡量。

\textbf{a. 周转时间}

周转时间是指作业从提交至完成的时间间隔,包括等待时间和执行时间;

周转时间Ti用公式表示为{作业i的周转时间Ti=作业i的完成时间-作业i的提交时间。}

\textbf{b.~平均周转时间}

平均周转时间是指多个作业(例如n个作业)周转时间的平均值;

平均周转时间用T表示为:{T=(T1+T2+\ldots{}\ldots{}+Tn)/n。}

\textbf{c. 带权周转时间}\\

带权周转时间是指作业周转时间与运行时间的比;

作业i的带权周转时间Wi表示为:{Wi=作业i的周转时间/作业i的运行时间。}

\textbf{d. 平均带权周转时间}

与平均周转时间类似,平均带权周转时间是多个作业的带权周转时间的平均值。
